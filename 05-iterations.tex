% LaTeX source for ``Python for Informatics: Exploring Information''
% Copyright (c)  2010-  Charles R. Severance, All Rights Reserved

\chapter{Iteración}
\index{iteración}

\section{Actualización de variables}
\label{update}

\index{actualización}
\index{variable!actualización}

Un uso habitual de las sentencias de asignación es aquel que consiste en
actualizar una variable --
donde el valor nuevo de la variable depende del antiguo.

\beforeverb
\begin{verbatim}
x = x+1
\end{verbatim}
\afterverb
%
Esto quiere decir ```toma el valor actual de {\tt x}, añádele 1, y luego
actualiza {\tt x} con el nuevo valor''.

Si intentas actualizar una variable que no existe, obtendrás
un error, ya que Python evalúa el lado derecho antes de asignar
el valor a {\tt x}:

\beforeverb
\begin{verbatim}
>>> x = x+1
NameError: name 'x' is not defined
\end{verbatim}
\afterverb
%
Antes de que puedas actualizar una variable, debes {\bf inicializarla},
normalmente mediante una simple asignación:

\index{inicialización (antes de actualizar)}

\beforeverb
\begin{verbatim}
>>> x = 0
>>> x = x+1
\end{verbatim}
\afterverb
%
Actualizar una variable añadiéndole 1 se denomina {\bf incrementar};
restarle 1 recibe el nombre de {\bf decrementar} (o disminuir).

\index{incrementar}
\index{decrementar}

\section{La sentencia {\tt while}}

\index{while, sentencia}
\index{sentencia!while}
\index{while, bucle}
\index{bucle!while}
\index{iteración}

Los ordenadores se suelen utilizar a menudo para automatizar tareas repetitivas. Repetir
tareas idénticas o muy similares sin cometer errores es algo que a los
ordenadores se les da bien y en cambio a las personas no.
Dado que esa interacción es muy común, Python proporciona varias
características en su lenguaje para hacerlo más sencillo.

Una forma de iteración en Python es la sentencia {\tt while}. He aquí un
programa sencillo que cuenta hacia atrás desde cinco y luego dice ``¡Despegue!''.

\beforeverb
\begin{verbatim}
n = 5
while n > 0:
    print n
    n = n-1
print '¡Despegue!'
\end{verbatim}
\afterverb
%
Casi se puede leer la sentencia {\tt while} como si estuviera escrita en inglés.
Significa, ``Mientras {\tt n} sea mayor que 0,
muestra el valor de {\tt n} y luego reduce el valor de {\tt n}
en 1 unidad. Cuando llegues a 0, sal de la sentencia {\tt while} y
muestra la palabra {\tt ¡Despegue!}''

\index{flujo de ejecución}

Éste es el flujo de ejecución de la sentencia {\tt while}, explicado de un modo más formal:

\begin{enumerate}

\item Se evalúa la condición, obteniendo {\tt Verdadero} or {\tt Falso}.

\item Si la condición es falsa, se sale de la sentencia {\tt while}
y se continúa la ejecución en la siguiente sentencia.

\item Si la condición es verdadera, se ejecuta el
cuerpo del {\tt while} y luego se vuelve al paso 1.

\end{enumerate}

Este tipo de flujo recibe el nombre de {\bf bucle}, ya que el tercer paso
enlaza de nuevo con el primero. Cada vez que se ejecuta el cuerpo del
bucle se dice que realizamos una {\bf iteración}. Para el bucle anterior,
podríamos decir que ``ha tenido cinco iteraciones'', lo que significa que el cuerpo
del bucle se ha ejecutado cinco veces.

\index{condición}
\index{bucle}
\index{cuerpo}

El cuerpo del bucle debe cambiar el valor de una o más variables,
de modo que la condición pueda en algún momento evaluarse como falsa
y el bucle termine.
La variable que cambia cada vez que el bucle se ejecuta
y controla cuándo termina éste, recibe el nombre de
{\bf variable de iteración}.
Si no hay variable de iteración, el bucle se repetirá para siempre,
resultando así un {\bf bucle infinito}.

\section{Bucles infinitos}

Una fuente de diversión sin fin para
los programadores es la constatación de que las instrucciones del champú:
``Enjabone, aclare, repita'', son un bucle infinito, ya que
no hay una {\bf variable de iteración} que diga cuántas veces
debe ejecutarse el proceso.

\index{infinito, bucle}
\index{bucle!infinito}

En el caso de una {\tt cuenta atrás}, podemos verificar que el bucle
termina, ya que sabemos que el valor de {\tt n} es finito, y podemos
ver que ese valor se va haciendo más pequeño cada vez que
se repite el bucle, de modo que en algún momento llegará a 0. Otras veces
un bucle es obviamente infinito, porque no tiene ninguna variable de iteración.

\section{``Bucles infinitos'' y {\tt break}}
\index{break, sentencia}
\index{sentencia!break}

A veces no se sabe si hay que terminar un bucle hasta que se ha
recorrido la mitad del cuerpo del mismo. En ese caso se puede crear un bucle infinito a propósito
y usar la sentencia {\tt break} para salir fuera del bucle cuando se desee.

El bucle siguiente es, obviamente, un {\bf bucle infinito}, porque la
expresión lógica de la sentencia
{\tt while} es simplemente la constante lógica {\tt True (verdadero)};

\beforeverb
\begin{verbatim}
n = 10
while True:
    print n, 
    n = n - 1
print '¡Terminado!'
\end{verbatim}
\afterverb
%
Si cometes el error de ejecutar este código, aprenderás rápidamente cómo
detener un proceso de Python bloqueado en el sistema, o tendrás que localizar dónde
se encuentra el botón de apagado de tu ordenador.
Este programa funcionará para siempre,
o hasta que la batería del equipo se termine,
ya que la expresión lógica al principio del bucle
es siempre cierta, en virtud del hecho de que esa expresión es
precisamente el valor constante {\tt True}.
 
A pesar de que en este caso se trata de un bucle infinito inútil, se puede usar ese diseño
para construir bucles útiles, siempre que se tenga la precaución de añadir código
en el cuerpo del bucle para salir explícitamente, usando {\tt break}
cuando se haya alcanzado la condición de salida.

Por ejemplo, supón que quieres recoger entradas de texto del usuario hasta que
éste escriba {\tt fin}. Podrías escribir:

\beforeverb
\begin{verbatim}
while True:
    linea = raw_input('> ')
    if linea == 'fin':
        break
    print linea
print '¡Terminado!'
\end{verbatim}
\afterverb
%
La condición del bucle es {\tt True}, lo cual es verdadero siempre, así que
el bucle se repetirá hasta que se ejecute la sentencia break.

Cada vez que se entre en el bucle, se pedirá una entrada al usuario.
Si el usuario escribe {\tt fin}, la sentencia {\tt break} hará que se
salga del bucle. En otro caso, el programa repetirá cualquier cosa que el usuario
escriba y volverá al principio del bucle. Éste es un ejemplo de su funcionamiento:

\beforeverb
\begin{verbatim}
> hola a todos
hola a todos
> he terminado
he terminado
> fin
¡Terminado!
\end{verbatim}
\afterverb
%
Este modo de escribir bucles {\tt while} es habitual, ya que
así se puede comprobar la condición en cualquier punto del
bucle (no sólo al principio), y se puede expresar la condición de parada
afirmativamente (``detente cuando ocurra''), en vez de tener que hacerlo con lógica negativa
(``sigue funcionando hasta que ocurra'').

\section{Terminando iteraciones con {\tt continue}}
\index{continue, sentencia}
\index{sentencia!continue}

Algunas veces, estando dentro de un bucle se necesita
terminar con la iteración actual y saltar a la siguiente de forma inmediata.
En ese caso se puede utilizar la sentencia
{\tt continue} para pasar a la siguiente iteración sin terminar
la ejecución del cuerpo del bucle para la actual.

A continuación se muestra un ejemplo de un bucle que repite lo que recibe como entrada hasta que
el usuario escribe ``fin'', pero trata las líneas que empiezan por el carácter almohadilla
como líneas que no deben mostrarse en pantalla (algo parecido a lo que hace Python con los comentarios).

\beforeverb
\begin{verbatim}
while True:
    linea = raw_input('> ')
    if linea[0] == '#' :
        continue
    if linea == 'fin':
        break
    print linea
print '¡Terminado!'
\end{verbatim}
\afterverb
%
He aquí una ejecución de ejemplo de este nuevo programa con la sentencia {\tt continue} añadida.

\beforeverb
\begin{verbatim}
> hola a todos
hola a todos
> # no imprimas esto
> ¡imprime esto!
¡imprime esto!
> fin
¡Terminado!
\end{verbatim}
\afterverb
%
Todas las líneas se imprimen en pantalla, excepto la que comienza con el símbolo
de almohadilla, ya que en ese caso se ejecuta {\tt continue}, finaliza
la iteración actual y salta de vuelta
a la sentencia {\tt while} para comenzar la siguiente iteración, de modo que
que se omite la sentencia {\tt print}.

\section{Bucles definidos usando {\tt for} }
\index{for, sentencia}
\index{sentencia!for}

A veces se desea repetir un bucle a través de un {\bf conjunto} de cosas, como
una lista de palabras, las líneas de un archivo, o una lista de números.
Cuando se tiene una lista de cosas para recorrer, se puede
construir un bucle \emph{definido} usando una sentencia {\tt for}.
A la sentencia {\tt while} se la llama un bucle \emph{indefinido},
porque simplemente se repite hasta que cierta condición se hace {\tt Falsa},
mientras que el bucle {\tt for} se repite a través de un
conjunto conocido de elementos, de modo que ejecuta tantas iteraciones como
elementos hay en el conjunto.

La sintaxis de un bucle {\tt for} es similar a la del bucle {\tt while},
en la cual hay una sentencia {\tt for} y un cuerpo que se repite:

\beforeverb
\begin{verbatim}
amigos = ['Joseph', 'Glenn', 'Sally']
for amigo in amigos:
    print 'Feliz año nuevo:', amigo
print '¡Terminado!'
\end{verbatim}
\afterverb
%
En términos de Python,
la variable {\tt amigos} es una lista\footnote{Examinaremos
las listas con más detalle en un capítulo posterior.}
de tres cadenas y el bucle {\tt for}
se mueve recorriendo la lista y ejecuta su cuerpo una vez
para cada una de las tres cadenas en la lista, produciendo
esta salida:

\beforeverb
\begin{verbatim}
Feliz año nuevo: Joseph
Feliz año nuevo: Glenn
Feliz año nuevo: Sally
¡Terminado!
\end{verbatim}
\afterverb
%

La traducción de este bucle {\tt for} al español no es tan directa como
en el caso del {\tt while}, pero si piensas en los amigos como un {\bf conjunto},
sería algo así como: ``Ejecuta las sentencias en el cuerpo del bucle
una vez para cada amigo que esté \emph{en (in)} el conjunto llamado amigos.''

Revisando el bucle, {\tt for}, {\bf for} e {\bf in} son palabras
reservadas de Python, mientras que {\tt amigo} y {\tt amigos} son variables.

{\tt {\bf for} amigo {\bf in} amigos{\bf :}\\
\verb"    "{\bf print} 'Feliz año nuevo', amigo }

En particular, {\tt amigo} es la {\bf variable de iteración} para
el bucle for. La variable {\tt amigo} cambia para cada iteración del
bucle y controla cuándo se termina el bucle {\tt for}. La
{\bf variable de iteracion} se desplaza sucesivamente a través de
las tres cadenas almacenadas en la variable {\tt amigos}.


\section{Diseños de bucles}

A menudo se usa un bucle {\tt for} o {\tt while} para movernos a través de una lista de elementos
o el contenido de un archivo y se busca algo, como el valor
más grande o el más pequeño de los datos que estamos revisando.

Los bucles generalmente se construyen así:

\begin{itemize}

\item Se inicializan una o más variables antes de que el bucle comience

\item Se realiza alguna operación con cada elemento en el cuerpo del bucle,
posiblemente cambiando las variables dentro de ese cuerpo.

\item Se revisan las variables resultantes cuando el bucle se completa

\end{itemize}

Usaremos ahora una lista de números para demostrar los conceptos y construcción
de estos diseños de bucles.

\subsection{Bucles de recuento y suma}

Por ejemplo, para contar el número de elementos
en una lista, podemos escribir el siguiente bucle {\tt for}:

\beforeverb
\begin{verbatim}
contador = 0
for valor in [3, 41, 12, 9, 74, 15]:
    contador = contador + 1
print 'Num. elementos: ', contador
\end{verbatim}
\afterverb
%
Ajustamos la variable {\tt contador} a cero antes de que el bucle comience,
después escribimos un bucle {\tt for} para movernos a través de la lista de números.
Nuestra variable de {\bf iteración} se llama {\tt valor}, y dado que no
usamos {\tt valor} dentro del bucle, lo único que hace es controlar el bucle
y hacer que el cuerpo del mismo sea ejecutado una vez para cada uno de los
valores de la lista.

En el cuerpo del bucle, añadimos 1 al valor actual de {\tt contador}
para cada uno de los valores de la lista. Mientras el bucle se está ejecutando, el
valor de {\tt contador} es el número de valores que se hayan visto ``hasta ese momento''.

Una vez el bucle se completa, el valor de {\tt contador} es el número total
de elementos. El número total ``cae en nuestro poder'' al final del
bucle. Se construye el bucle de modo que obtengamos lo que queremos cuando el
bucle termina.

Otro bucle similar, que calcula el total de un conjunto de números,
se muestra a continuación:

\beforeverb
\begin{verbatim}
total = 0
for valor in [3, 41, 12, 9, 74, 15]:
    total = total + valor
print 'Total: ', total
\end{verbatim}
\afterverb
%
En este bucle, \emph{sí} utilizamos la {\bf variable de iteración}.
En vez de añadir simplemente uno a {\tt contador} como en el bucle previo,
ahora durante cada iteración del bucle añadimos el número actual (3, 41, 12, etc.)
al total en ese momento.
Si piensas en la variable {\tt total}, ésta contiene la
``suma parcial de valores hasta ese momento''. Así que antes de que el bucle
comience, {\tt total} es cero, porque aún no se ha examinado ningún valor.
Durante el bucle, {\tt total} es la suma parcial, y al final del bucle,
{\tt total} es la suma total definitiva de todos los valores
de la lista.

Cuando el bucle se ejecuta, {\tt total} acumula la suma de los elementos;
una variable que se usa de este modo recibe a veces el nombre de
{\bf acumulador}.
\index{acumulador!sum}

Ni el bucle que cuenta los elementos ni el que los suma resultan particularmente
útiles en la práctica, dado que existen las funciones internas
{\tt len()} y {\tt sum()} que cuentan el número de elementos
de una lista y el total de elementos en la misma
respectivamente.

\subsection{Bucles de máximos y mínimos}

\index{bucle!máximo}
\index{bucle!mínimo}
\index{None, valor especial}
\index{valor especial!None}
\label{maximumloop}
Para encontrar el valor mayor de una lista o secuencia, construimos
el bucle siguiente:

\beforeverb
\begin{verbatim}
mayor = None
print 'Antes:', mayor
for valor in [3, 41, 12, 9, 74, 15]:
    if mayor is None or valor > mayor :
        mayor = valor
    print 'Bucle:', valor, mayor
print 'Mayor:', mayor
\end{verbatim}
\afterverb
%
Cuando se ejecuta el programa, se obtiene la siguiente salida:

\beforeverb
\begin{verbatim}
Antes: None
Bucle: 3 3
Bucle: 41 41
Bucle: 12 41
Bucle: 9 41
Bucle: 74 74
Bucle: 15 74
Mayor: 74
\end{verbatim}
\afterverb
%
Debemos pensar en la variable {\tt mayor} como
el ``mayor valor visto hasta ese momento''.
Antes del bucle, asignamos a {\tt mayor} el valor {\tt None}.
{\tt None} es un valor constante especial que se puede
almacenar en una variable para indicar
que la variable está ``vacía''.

Antes de que el bucle comience, el mayor valor visto hasta entonces
es {\tt None}, dado que no se ha visto aún ningún valor. Durante la
ejecución del bucle, si {\tt mayor} es {\tt None}, entonces
tomamos el primer valor que hemos visto como el mayor hasta entonces. Se puede ver en
la primera iteración, cuando el valor de {\tt valor} es 3,
mientras que {\tt mayor} es {\tt None}, inmediatamente
{\tt mayor} pasa a ser 3.

Tras la primera iteración, {\tt mayor} ya no es {\tt None},
así que la segunda parte de la expresión lógica compuesta que comprueba
si {\tt valor > mayor} se activará sólo cuando encontremos un valor que es
mayor que el ``mayor hasta ese momento''. Cuando encontramos un nuevo valor ``mayor aún'',
tomamos ese nuevo valor para {\tt mayor}. Se puede ver en la salida
del programa que {\tt mayor} pasa desde 3 a 41 y luego a 74.

Al final del bucle, se habrán revisado todos los valores y la
variable {\tt mayor} contendrá entonces el mayor valor de
la lista.

Para calcular el número más pequeño, el código es muy similar con un
pequeño cambio:

\beforeverb
\begin{verbatim}
menor = None
print 'Antes:', menor
for valor in [3, 41, 12, 9, 74, 15]:
    if menor is None or valor < menor:
        menor = valor
    print 'Bucle:', valor, menor
print 'Menor:', menor
\end{verbatim}
\afterverb
%
De nuevo, {\tt menor} es el ``menor hasta ese momento'' antes, durante y después de
que el bucle se ejecute. Cuando el bucle se ha completado, {\tt menor} contendrá
el mínimo valor de la lista

También como en el caso del número de elementos y de la suma, las funciones internas
{\tt max()} y {\tt min()} convierten la escritura de este tipo de bucles
en innecesaria.

Lo siguiente es una versión simple de la función interna de Python
{\tt min()}:

\beforeverb
\begin{verbatim}
def min(valores):
    menor = None
    for valor in valores:
        if menor is None or valor < menor:
            menor = valor
    return menor
\end{verbatim}
\afterverb
%
En esta versión de la función para calcular el mínimo, hemos eliminado las
sentencias {\tt print}, de modo que sea equivalente a la función {\tt min},
que ya está incorporada dentro de Python.

\section{Depuración}
\index{depuración}

A medida que vayas escribiendo programas más grandes, puede que notes que vas necesitando
emplear cada vez más tiempo en depurarlos. Más código significa más oportunidades de
cometer un error y más lugares para que los bugs puedan esconderse.

\index{depuración!por bisección}
\index{bisección, depuración por}

Un método para acortar el tiempo de depuración es ``depurar por bisección''.
Por ejemplo, si hay 100 líneas en tu programa y las compruebas
de una en una, te llevará 100 pasos.

En lugar de eso, intenta partir el problema por la mitad. Busca en medio
del programa, o cerca de ahí, un valor intermedio que puedas
comprobar. Añade una sentencia {\tt print} (o alguna otra cosa
que tenga un efecto verificable), y haz funcionar el programa.

Si en el punto medio la verificación es incorrecta, el problema debería
estar en la primera mitad del programa. Si ésta es correcta, el problema
estará en la segunda mitad.

Cada vez que realices una comprobación como esta, reduces a la mitad el número
de líneas en las que buscar. Después de seis pasos (que son muchos
menos de 100), lo habrás reducido a una o dos líneas de código,
al menos en teoría.

En la práctica no siempre está claro qué es
``el medio del programa'', y no siempre es posible colocar ahí
una verificación. No tiene sentido contar las líneas y encontrar
el punto medio exacto. En lugar de eso, piensa en lugares del programa
en los cuales pueda haber errores y en lugares donde sea fácil colocar una verificación.
Luego elige un lugar donde creas que las oportunidades de que el bug
esté por delante y las de que esté por detrás son más o menos las mismas.

\section{Glosario}

\begin{description}

\item[acumulador:] Una variable usada en un bucle para sumar o
acumular un resultado.
\index{acumulador}

\item[bucle infinito:] Un bucle en el cual la condición de terminación no
se satisface nunca o para el cual no existe dicha condición de terminación.
\index{infinito, bucle}
\index{bucle!infinito}

\item[contador:] Una variable usada en un bucle para contar el número
de veces que algo sucede. Inicializamos el contador a
cero y luego lo vamos incrementando cada vez que queramos que
``cuente'' algo.
\index{contador}

\item[decremento:] Una actualización que disminuye el valor de una variable.
\index{decremento}

\item[inicializar:] Una asignación que da un valor inicial a
una variable que va a ser después actualizada.
\index{inicializar!variable}

\item[incremento:] Una actualización que aumenta el valor de una variable
(a menudo en una unidad).
\index{incremento}

\item[iteración:] Ejecución repetida de una serie de sentencias usando
bien una función que se llama a si misma o bien un bucle.
\index{iteración}

\end{description}


\section{Ejercicios}

\begin{ex}
Escribe un programa que lea repetidamente números hasta que el usuario
introduzca ``fin''.
Una vez se haya introducido ``fin'', muestra por pantalla el total, la cantidad de números y la media
de esos números. Si el usuario introduce cualquier otra cosa que no sea un número,
detecta su error usando {\tt try} y {\tt except},
muestra un mensaje de error y pasa al número siguiente.

\begin{verbatim}
Introduce un número: 4
Introduce un número: 5
Introduce un número: dato erróneo
Entrada inválida
Introduce un número: 7
Introduce un número: fin
16 3 5.33333333333
\end{verbatim}
\end{ex}

\begin{ex}
Escribe otro programa que pida una lista de números como la anterior
y al final muestre por pantalla el máximo y mínimo de los números, en vez de la media.
\end{ex}


