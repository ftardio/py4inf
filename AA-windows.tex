% The contents of this file is 
% Copyright (c) 2009-  Charles R. Severance, All Righs Reserved

\chapter{Programando con Python en Windows}

En este apéndice, mostraremos una serie de pasos
para que puedas ejecutar Python en Windows. Existen muchos métodos
diferentes que se pueden seguir, y éste es sólo uno de ellos
que intenta hacer las cosas de una forma sencilla.

Lo primero que necesitas es instalar un editor de código. No
querrás utilizar Notepad o Microsoft Word para editar
programas en Python. Los programas deben estar en ficheros de ``texto plano'',
de modo que necesitas un editor que sea capaz de
editar archivos de texto.

Nuestra recomendación como editor para Windows es NotePad++, que
puede descargarse e instalarse desde:

\url{https://notepad-plus-plus.org/}

Luego descarga una versión actual de Python 2 desde el
sitio web \url{www.python.org}.

\url{https://www.python.org/downloads/}

Una vez hayas instalado Python, deberías tener una
carpeta nueva en tu equipo como {\tt C:{\textbackslash}Python27}.

Para crear un programa en Python, ejecuta NotePad++ desde el menú de Inicio de Windows
y guarda el fichero con la extensión ``.py''. Para este
ejercicio, crea una carpeta en tu Escritorio llamada
{\tt p4inf}. Es mejor usar nombres de carpeta cortos
y no utilizar espacios en los nombres de carpetas ni de archivos.

Vamos a hacer que nuestro primer programa en Python sea:

\beforeverb
\begin{verbatim}
print 'Hola, Chuck'
\end{verbatim}
\afterverb
%
Excepto que deberías cambiarlo para que escriba tu nombre. Guarda el fichero
en {\tt Escritorio{\textbackslash}py4inf{\textbackslash}prog1.py}.

Luego abre una ventana de línea de comandos. En cada versión de Windows
se hace de una forma diferente:

\begin{itemize}
\item Windows 10: Teclea {\tt command} en el cuadro de búsqueda
que se encuentra en la barra de tareas, en la parte inferior del Escritorio,
y pulsa intro.
	
\item Windows Vista y Windows 7: Pulsa el botón de {\bf Inicio}
y luego en la ventana de búsqueda de comandos introduce la palabra
{\tt command} y pulsa intro.

\item Windows XP: Pulsa el botón de {\bf Inicio}, luego {\bf Ejecutar}, y
a continuación introduce {\tt cmd} en la ventana de diálogo y pulsa {\bf OK}.
\end{itemize}

Te encontrarás en una ventana de texto con un indicador que
te dice en qué carpeta estás actualmente ubicado.

Windows Vista y Windows-7-10: {\tt C:{\textbackslash}Users{\textbackslash}csev}\\
Windows XP: {\tt C:{\textbackslash}Documents and Settings{\textbackslash}csev}

Éste es tu ``directorio de inicio'' ({\tt home}). Ahora tenemos que movernos hasta
la carpeta donde hemos guardado nuestro programa Python usando
los siguientes comandos:

\beforeverb
\begin{verbatim}
C:\Users\csev\> cd Desktop
C:\Users\csev\Desktop> cd py4inf
\end{verbatim}
\afterverb
%
Luego teclea

\beforeverb
\begin{verbatim}
C:\Users\csev\Desktop\py4inf> dir 
\end{verbatim}
\afterverb
%
para mostrar un listado de tus archivos. Al hacerlo, 
deberías ver el archivo {\tt prog1.py}.

Para ejecutar tu programa, simplemente teclea el nombre del fichero en
el indicador de comando y pulsa intro.

\beforeverb
\begin{verbatim}
C:\Users\csev\Desktop\py4inf> prog1.py
Hola, Chuck
C:\Users\csev\Desktop\py4inf> 
\end{verbatim}
\afterverb
%
Puedes editar el fichero en NotePad++, guardarlo, y luego volver
a la línea de comandos y ejecutarlo otra vez, tecleando de nuevo
el nombre del fichero en el indicador.

Si te has perdido en la ventana de la línea de comandos, tan solo tienes
que cerrarla y abrir una nueva.

Pista: También puedes pulsar la tecla ``flecha arriba'' en la línea de comandos para
desplazarte hacia atrás y ejecutar de nuevo un comando introducido anteriormente.

Además, deberías buscar en las preferencias de NotePad++, y ajustarlas para
que sustituya los caracteres de tabulación por cuatro espacios. Esto te ahorrará
un montón de esfuerzo a la hora de localizar errores de justificación en el código.

Puedes encontrar más información sobre la edición y ejecución de
programas en Python en \url{www.py4inf.com}.

