% LaTeX source for ``Python for Informatics: Exploring Information''
% Copyright (c)  2010-  Charles R. Severance, All Rights Reserved

\chapter{Variables, expresiones y sentencias}

\section{Valores y tipos}
\index{valor}
\index{tipo}
\index{cadena}

Un {\bf valor} es una de las cosas básicas que utiliza un programa,
como una letra o un número.
Los valores que hemos visto hasta ahora
han sido {\tt 1}, {\tt 2}, y
\verb"'¡Hola, mundo!'"

Esos valores pertenecen a {\bf tipos} diferentes:
{\tt 2} es un entero (int), y \verb"'¡Hola, mundo!'" es una {\bf cadena} (string),
que recibe ese nombre porque contiene una ``cadena'' de letras.
Tú (y el intérprete) podéis identificar
las cadenas porque van encerradas entre comillas.

\index{comillas}

La sentencia {\tt print} también funciona con enteros. Vamos a usar el
comando {\tt python} para iniciar el intérprete.

\beforeverb
\begin{verbatim}
python
>>> print 4
4
\end{verbatim}
\afterverb
%
Si no estás seguro de qué tipo de valor estás manejando, el intérprete te lo puede decir.

\beforeverb
\begin{verbatim}
>>> type('¡Hola, mundo!')
<type 'str'>
>>> type(17)
<type 'int'>
\end{verbatim}
\afterverb
%
No resulta sorprendente que las cadenas pertenezca al tipo {\tt str}, y los
enteros pertenezcan al tipo {\tt int}. Resulta, sin embargo, menos obvio
que los números con un punto decimal pertenezcan a un tipo llamado {\tt float} (flotante),
debido a que esos números se representan en un formato
conocido como {\bf punto flotante}\footnote{En el mundo anglosajón (y también en Python)
la parte decimal de un número se separa de la parte entera mediante un punto, y no mediante una coma
(Nota del trad.)}.

\index{tipo}
\index{cadena!tipo}
\index{tipo!str}
\index{int, tipo}
\index{tipo!int}
\index{float, tipo}
\index{tipo!float}

\beforeverb
\begin{verbatim}
>>> type(3.2)
<type 'float'>
\end{verbatim}
\afterverb
%
¿Qué ocurre con valores como \verb"'17'" y \verb"'3.2'"?
Parecen números, pero van entre comillas como
las cadenas.

\index{comillas}

\beforeverb
\begin{verbatim}
>>> type('17')
<type 'str'>
>>> type('3.2')
<type 'str'>
\end{verbatim}
\afterverb
%
Son cadenas.

Cuando escribes un entero grande, puede que te sientas tentado a usar comas
o puntos para separarlo en grupos de tres dígitos, como en {\tt 1,000,000}
\footnote{En el mundo anglosajón el ``separador de millares'' es la coma, y no el punto (Nota del trad.)}.
Eso no es un entero válido en Python, pero en cambio sí que resulta válido algo como:

\beforeverb
\begin{verbatim}
>>> print 1,000,000
1 0 0
\end{verbatim}
\afterverb
%
Bien, ha funcionado. ¡Pero eso no era lo que esperábamos!. Python interpreta
{\tt 1,000,000} como una secuencia de enteros separados por comas, así que lo
imprime con espacios en medio.

\index{semántico, error}
\index{error!semántico}
\index{mensaje de error}

Éste es el primer ejemplo que hemos visto de un error semántico: el código
funciona sin producir ningún mensaje de error, pero no hace su trabajo
``correctamente''.

\section{Variables}
\index{variable}
\index{asignación!sentencia}
\index{sentencia!asignación}

Una de las características más potentes de un lenguaje de programación es
la capacidad de manipular {\bf variables}. Una variable es un nombre
que se refiere a un valor.

Una {\bf sentencia de asignación} crea variables nuevas y las
da valores:

\beforeverb
\begin{verbatim}
>>> mensaje = 'Y ahora algo completamente diferente'
>>> n = 17
>>> pi = 3.1415926535897931
\end{verbatim}
\afterverb
%
Este ejemplo hace tres asignaciones. La primera asigna una cadena
a una variable nueva llamada {\tt mensaje};
la segunda asigna el entero {\tt 17} a {\tt n}; la tercera
asigna el valor (aproximado) de $\pi$ a {\tt pi}.

Para mostrar el valor de una variable, se puede usar la sentencia print:

\beforeverb
\begin{verbatim}
>>> print n
17
>>> print pi
3.14159265359
\end{verbatim}
\afterverb
%
El tipo de una variable es el tipo del valor al que se refiere.

\beforeverb
\begin{verbatim}
>>> type(mensaje)
<type 'str'>
>>> type(n)
<type 'int'>
>>> type(pi)
<type 'float'>
\end{verbatim}
\afterverb
%

\section{Nombres de variables y palabras claves}
\index{palabra clave}

Los programadores generalmente eligen nombres para sus variables que
tengan sentido y documenten para qué se usa esa variable.

Los nombres de las variables pueden ser arbitrariamente largos. Pueden contener
tanto letras como números, pero no pueden comenzar con un número.
Se pueden usar letras mayúsculas, pero es buena idea
comenzar los nombres de las variables con una letras minúscula (veremos
por qué más adelante).

El carácter guión-bajo (\verb"_") puede utilizarse en un nombre.
A menudo se utiliza en nombres con múltiples palabras, como en
\verb"mi_nombre" o \verb"velocidad_de_golondrina_sin_carga".
Los nombres de las variables pueden comenzar con un carácter guión-bajo, pero
generalmente se evita usarlo así a menos que se esté escribiendo código
para librerías que luego utilizarán otros.

\index{guión-bajo, carácter}

Si se le da a una variable un nombre no permitido, se obtiene un error de sintaxis:

\beforeverb
\begin{verbatim}
>>> 76trombones = 'gran desfile'
SyntaxError: invalid syntax
>>> more@ = 1000000
SyntaxError: invalid syntax
>>> class = 'Teorema avanzado de Zymurgy'
SyntaxError: invalid syntax
\end{verbatim}
\afterverb
%
{\tt 76trombones} es incorrecto porque comienza por un número.
{\tt more@} es incorrecto porque contiene un carácter no premitido, {\tt @}.
Pero, ¿qué es lo que está mal en {\tt class}?

Pues resulta que {\tt class} es una de las {\bf palabras clave} de Python. El
intérprete usa palabras clave para reconocer la estructura del programa,
y esas palabras no pueden ser utilizadas como nombres de variables.

\index{palabra clave}

Python reserva 31 palabras claves\footnote{En Python 3.0, {\tt exec} ya no es
una palabra clave, pero {\tt nonlocal} sí que lo es.} para su propio uso:

\beforeverb
\begin{verbatim}
and       del       from      not       while    
as        elif      global    or        with     
assert    else      if        pass      yield    
break     except    import    print              
class     exec      in        raise              
continue  finally   is        return             
def       for       lambda    try
\end{verbatim}
\afterverb
%
Puede que quieras tener esta lista a mano. Si el intérprete se queja
por el nombre de una de tus variables y no sabes por qué, comprueba
si ese nombre está en esta lista.

\section{Sentencias}

Una {\bf sentencia} es una unidad de código que el intérprete de Python puede
ejecutar. Hemos visto hasta ahora dos tipos de sentencia: print
y las asignaciones.

\index{sentencia}
\index{interactivo, modo}
\index{script, modo}

Cuando escribes una sentencia en modo interactivo, el intérprete la
ejecuta y muestra el resultado, si es que lo hay.

Un script normalmente contiene una secuencia de sentencias. Si hay
más de una sentencia, los resultados aparecen de uno en uno
según se van ejecutando las sentencias.

Por ejemplo, el script

\beforeverb
\begin{verbatim}
print 1
x = 2
print x
\end{verbatim}
\afterverb
%
produce la salida

\beforeverb
\begin{verbatim}
1
2
\end{verbatim}
\afterverb
%
La sentencia de asignación no produce ninguna salida.


\section{Operadores y operandos}
\index{operador!aritmético}
\index{aritmético, operador}
\index{operando}
\index{expresión}

{\bf Los operadores} son símbolos especiales que representan cálculos, como
la suma o la multiplicación. Los valores a los cuales se aplican esos operadores
reciben el nombre de {\bf operandos}.

Los operadores {\tt +}, {\tt -}, {\tt *}, {\tt /}, y {\tt **}
realizan sumas, restas, multiplicaciones, divisiones y
exponenciación (elevar un número a una potencia), como se muestra en los ejemplos siguientes:

\beforeverb
\begin{verbatim}
20+32   hora-1   hora*60+minuto   minuto/60   5**2   (5+9)*(15-7)
\end{verbatim}
\afterverb
%
El operador de división puede que no haga exactamente lo que esperas:

\beforeverb
\begin{verbatim}
>>> minuto = 59
>>> minuto/60
0
\end{verbatim}
\afterverb
%
El valor de {\tt minuto} es 59, y en la aritmética convencional 59
dividido por 60 es 0.98333, no 0. La razón de esta discrepancia es
que Python está realizando {\bf división entera}\footnote{En Python 3.0,
el resultado de esta división es un número {\tt flotante}.
En Python 3.0, el nuevo operador
{\tt //} es el que realiza la división entera.}.

\index{Python 3.0}
\index{entera, división}
\index{punto-flotante, división}
\index{división!entera}
\index{división!punto-flotante}

Cuando ambos operandos son enteros, el resultado es también un
entero; la división entera descarta la parte decimal,
así que en este ejemplo trunca la respuesta a cero.

Si cualquiera de los operandos es un número en punto flotante, Python realiza
división en punto flotante, y el resultado es un {\tt float}:

\beforeverb
\begin{verbatim}
>>> minuto/60.0
0.98333333333333328
\end{verbatim}
\afterverb


\section{Expresiones}

Una {\bf expresión} es una combinación de valores, variables y operadores.
Un valor por si mismo se considera una expresión, y también lo es
una variable, así que las siguientes expresiones son todas válidas
(asumiendo que la variable {\tt x} tenga un valor asignado):

\index{expresión}
\index{evaluar}

\beforeverb
\begin{verbatim}
17
x
x + 17
\end{verbatim}
\afterverb
%
Si escribes una expresión en modo interactivo, el intérprete la
{\bf evalúa} y muestra el resultado:

\beforeverb
\begin{verbatim}
>>> 1 + 1
2
\end{verbatim}
\afterverb
%
Sin embargo, en un script, ¡una expresión por si misma no hace nada!
Esto a menudo puede producir confusión
entre los principiantes.

\begin{ex}
Escribe las siguientes sentencias en el intérprete de Python para comprobar
qué hacen:

\beforeverb
\begin{verbatim}
5
x = 5
x + 1
\end{verbatim}
\afterverb
%
\end{ex}


\section{Orden de las operaciones}
\index{orden de operaciones}
\index{reglas de precedencia}
\index{PEMDSR}

Cuando en una expresión aparece más de un operador, el orden de
evaluación depende de las {\bf reglas de precedencia}. Para los
operadores matemáticos, Python sigue las convenciones matemáticas.
El acrónimo {\bf PEMDSR} resulta útil para
recordar esas reglas:

\index{paréntesis!invalidar precedencia}

\begin{itemize}

\item Los {\bf P}aréntesis tienen el nivel superior de precedencia, y pueden usarse
para forzar a que una expresión sea evaluada en el orden que se quiera. Dado
que las expresiones entre paréntesis son evaluadas primero, {\tt 2 * (3-1)} es 4,
y {\tt (1+1)**(5-2)} es 8. Se pueden usar también paréntesis para hacer una
expresión más sencilla de leer, incluso si el resultado de la misma no varía por ello,
como en {\tt (minuto * 100) / 60}.

\item La {\bf E}xponenciación (elevar un número a una potencia) tiene el siguiente nivel más alto de
precedencia, de modo que {\tt 2**1+1} es 3, no 4, y {\tt 3*1**3} es 3, no 27.

\item La {\bf M}ultiplicación y la {\bf D}ivisión tienen la misma precedencia,
que es superior a la de la {\bf S}uma y la {\bf R}esta, que también tienen entre si el mismo
nivel de precedencia. Así que {\tt 2*3-1} es 5, no 4, y
{\tt 6+4/2} es 8, no 5.

\item Los operadores con igual precedencia son evaluados de izquierda a
derecha. Así que la expresión {\tt 5-3-1} es 1 y no 3, ya que
{\tt 5-3} se evalúa antes, y después se resta {\tt 1} de {\tt 2}.

\end{itemize}

En caso de duda, añade siempre paréntesis a tus expresiones para asegurarte
de que las operaciones se realizan en el orden que tú quieres.

\section{Operador módulo}

\index{módulo, operador}
\index{operador!módulo}

El {\bf operador módulo} trabaja con enteros y obtiene el resto
de la operación consistente en dividir el primer operando por el segundo. En Python, el
operador módulo es un signo de porcentaje (\verb"%"). La sintaxis es la misma
que se usa para los demás operadores:

\beforeverb
\begin{verbatim}
>>> cociente = 7 / 3
>>> print cociente
2
>>> resto = 7 % 3
>>> print resto
1
\end{verbatim}
\afterverb
%
Así que 7 dividido por 3 es 2 y nos sobra 1. 

El operador módulo resulta ser sorprendentemente útil. Por ejemplo,
puedes comprobar si un número es divisible por otro---si
{\tt x \% y} es cero, entonces {\tt x} es divisible por {\tt y}.

\index{divisibilidad}

También se puede extraer el dígito más a la derecha
de los que componen un número. Por ejemplo, {\tt x \% 10} obtiene el
dígito que está más a la derecha de {\tt x} (en base 10). De forma similar, {\tt x \% 100}
obtiene los dos últimos dígitos.

\section{Operaciones con cadenas}
\index{cadena!operación}
\index{operador!cadena}

El operador {\tt +} funciona con las cadenas, pero
no realiza una suma en el sentido matemático. En vez de eso, realiza una
{\bf concatenación}, que quiere decir que une ambas cadenas,
enlazando el final de la primera con el principio de la segunda. Por ejemplo:

\index{concatenación}

\beforeverb
\begin{verbatim}
>>> primero = 10
>>> segundo = 15
>>> print primero+segundo
25
>>> primero = '100'
>>> segundo = '150'
>>> print primero + segundo
100150
\end{verbatim}
\afterverb
%
La salida de este programa es {\tt 100150}.

\section{Petición de información al usuario}
\index{entrada desde teclado}

A veces necesitaremos que sea el usuario quien nos proporcione el valor para
una variable, a través del teclado.
Python proporciona una función interna llamada \verb"raw_input" que recibe
la entrada desde el teclado\footnote{En Python 3.0, esta función ha sido llamada
{\tt input}.}. Cuando se llama a esa función, el programa se detiene y
espera a que el usuario escriba algo. Cuando el usuario pulsa {\sf Retorno} o
{\sf Intro}, el programa continúa y \verb"raw_input"
devuelve como una cadena aquello que el usuario escribió.

\index{Python 3.0}
\index{raw\_input, función}
\index{función!raw\_input}

\beforeverb
\begin{verbatim}
>>> entrada = raw_input()
Cualquier cosa ridícula
>>> print entrada
Cualquier cosa ridícula
\end{verbatim}
\afterverb
%
Antes de recibir cualquier dato desde el usuario, es buena idea escribir
un mensaje explicándole qué debe introducir. Se puede pasar una cadena a
\verb"raw_input", que será mostrada al usuario antes de que el programa se detenga
para recibir su entrada:

\index{indicador}

\beforeverb
\begin{verbatim}
>>> nombre = raw_input('¿Cómo te llamas?\n')
¿Cómo te llamas?
Chuck
>>> print nombre
Chuck
\end{verbatim}
\afterverb
%
La secuencia \verb"\n" al final del mensaje representa un {\bf newline},
que es un carácter especial que provoca un salto de línea.
Por eso la entrada del usuario aparece debajo de nuestro mensaje.

\index{salto de línea}

Si esperas que el usuario escriba un entero, puedes intentar convertir
el valor de retorno a {\tt int} usando la función {\tt int()}:

\beforeverb
\begin{verbatim}
>>> prompt = '¿Cual.... es la velocidad de vuelo de una golondrina sin carga?\n'
>>> velocidad = raw_input(prompt)
¿Cual.... es la velocidad de vuelo de una golondrina sin carga?
17
>>> int(velocidad)
17
>>> int(velocidad) + 5
22
\end{verbatim}
\afterverb
%
Pero si el usuario escribe algo que no sea una cadena de dígitos,
obtendrás un error:

\beforeverb
\begin{verbatim}
>>> velocidad = raw_input(prompt)
¿Cual.... es la velocidad de vuelo de una golondrina sin carga?
¿Te refieres a una golondrina africana o a una europea?
>>> int(velocidad)
ValueError: invalid literal for int()
\end{verbatim}
\afterverb
%
Veremos cómo controlar este tipo de errores más adelante.

\index{ValueError}
\index{exception!ValueError}


\section{Comentarios}
\index{comentarios}

A medida que los programas se van volviendo más grandes y complicados, se vuelven más difíciles
de leer. Los lenguajes formales son densos, y a menudo es complicado
mirar un trozo de código e imaginarse qué es lo que hace, o por qué.

Por eso es buena idea añadir notas a tus programas, para explicar
en un lenguaje normal qué es lo que el programa está haciendo. Estas notas reciben el nombre de
{\bf comentarios}, y en Python comienzan con el símbolo \verb"#":


\beforeverb
\begin{verbatim}
# calcula el porcentaje de hora transcurrido
porcentaje = (minuto * 100) / 60
\end{verbatim}
\afterverb
%
En este caso, el comentario aparece como una línea completa. Pero también puedes
poner comentarios al final de una línea

\beforeverb
\begin{verbatim}
porcentaje = (minuto * 100) / 60     # porcentaje de una hora
\end{verbatim}
\afterverb
%
Todo lo que va desde {\tt \#} hasta el final de la línea es ignorado---no
afecta para nada al programa.

Las comentarios son más útiles cuando documentan características del código
que no resultan obvias. Es razonable asumir que el lector puede descifrar
\emph{qué} es lo que el código hace; es mucho más útil explicarle \emph{por qué}.

Este comentario es redundante con el código e inútil:

\beforeverb
\begin{verbatim}
v = 5     # asigna 5 a v
\end{verbatim}
\afterverb
%
Este comentario contiene información útil que no está en el código:

\beforeverb
\begin{verbatim}
v = 5     # velocidad en metros/segundo. 
\end{verbatim}
\afterverb
%
Elegir nombres adecuados para las variables puede reducir la necesidad de comentarios, pero
los nombres largos también pueden ocasionar que las expresiones complejas sean difíciles de leer, así
que hay que conseguir una solución de compromiso.

\section{Elección de nombres de variables mnemónicos}

\index{mnemónico}

Mientras sigas las sencillas reglas de nombrado de variables y evites las
palabras reservadas, dispondrás de una gran variedad de opciones para poner nombres a tus variables.
Al principio, esa diversidad puede llegar a resultarte confusa, tanto al leer un programa
como al escribir el tuyo propio. Por ejemplo, los tres
programas siguientes son idénticos en cuanto a la función que realizan,
pero muy diferentes cuando los lees e intentas entenderlos.

\beforeverb
\begin{verbatim}
a = 35.0
b = 12.50
c = a * b
print c

horas = 35.0
tarifa = 12.50
salario = horas * tarifa
print salario

x1q3z9ahd = 35.0
x1q3z9afd = 12.50
x1q3p9afd = x1q3z9ahd * x1q3z9afd
print x1q3p9afd
\end{verbatim}
\afterverb
%
El intérprete de Python ve los tres programas como \emph{exactamente idénticos},
pero los humanos ven y asimilan estos programas de forma bastante diferente.
Los humanos entenderán más rápidamente el {\bf objetivo}
del segundo programa, ya que el programador
ha elegido nombres de variables que reflejan lo que pretendía
de acuerdo al contenido que iba almacenar en cada variable.

Esa sabia elección de nombres de variables se denomina utilizar ``nombres de variables mnemónicos''.
La palabra \emph{mnemónico}\footnote{Consulta
\url{https://es.wikipedia.org/wiki/Mnemonico}
para obtener una descripción detallada de la palabra ``mnemónico''.}
significa ``que ayuda a memorizar''.
Elegimos nombres de variables mnemónicos para ayudarnos a recordar por qué creamos las variables
al principio.

A pesar de que todo esto parezca estupendo, y de que sea una idea muy buena usar nombres
de variables mnemónicos, ese tipo de nombres pueden interponerse en el camino de los programadores
novatos a la hora de analizar y comprender el código. Esto se debe a que los programadores
principiantes no han memorizado aún las palabras reservadas (sólo hay 31), y a veces
variables con nombres que son demasiado descriptivos pueden llegar a parecerles
parte del lenguaje y no simplemente nombres de variable bien elegidos\footnote{El párrafo anterior
se refiere más bien a quienes eligen nombres de variables en inglés, ya que todas las
palabras reservadas de Python coinciden con palabras propias de ese idioma (Nota del trad.)}.

Echa un vistazo rápido al siguiente código de ejemplo en Python, que se mueve en bucle a través de
un conjunto de datos. Trataremos los bucles pronto, pero por ahora tan sólo trata de entender
su significado:

\beforeverb
\begin{verbatim}
for word in words:
    print word
\end{verbatim}
\afterverb
%
¿Qué ocurre aquí? ¿Cuáles de las piezas (for, word, in, etc.) son palabras reservadas
y cuáles son simplemente nombres de variables? ¿Acaso Python comprende de un modo
básico la noción de palabras ({\tt words})? Los programadores novatos tienen
problemas separando qué parte del código
\emph{debe} mantenerse tal como está en este ejemplo y qué partes son simplemente
elección del programador.

El código siguiente es equivalente al de arriba:

\beforeverb
\begin{verbatim}
for porcion in pizza:
    print porcion
\end{verbatim}
\afterverb
%
Para los principiantes es más fácil estudiar este código y saber qué partes
son palabras reservadas definidas por Python y qué partes son simplemente nombres
de variables elegidas por el programador. Está bastante claro que Python no
entiende nada de pizza ni de porciones,
ni del hecho de que una pizza consiste en un conjunto de una o más porciones.

Pero si nuestro programa lo que realmente va a hacer es leer datos y buscar palabras en ellos,
{\tt pizza} y {\tt porción} son nombres muy poco mnemónicos. Elegirlos como
nombres de variables distrae del propósito real del programa.

Dentro de muy poco tiempo, conocerás las palabras reservadas más comunes,
y empezarás a ver cómo esas palabras reservadas resaltan sobre las demás:

{\tt {\bf for} word {\bf in} words{\bf :}\\
\verb"    "{\bf print} word }

Las partes del código que están definidas por
Python ({\tt for}, {\tt in}, {\tt print}, y {\tt :}) están en negrita,
mientras que las variables elegidas por el programador ({\tt word} y {\tt words}) no lo están.
Muchos editores de texto son conscientes de la sintaxis de Python
y colorearán las palabras reservadas de forma diferente para darte pistas que te permitan
mantener tus variables y las palabras reservadas separados.
Dentro de poco empezarás a leer Python y podrás determinar rápidamente qué
es una variable y qué es una palabra reservada.

\section{Depuración}
\index{depuración}

En este punto, el error de sintaxis que es más probable que cometas será
intentar utilizar nombres de variables no válidos, como {\tt class} y {\tt yield}, que
son palabras clave, o \verb"odd~job" y \verb"US$", que contienen
caracteres no válidos.

\index{sintaxis, error}
\index{error!sintaxis}

Si pones un espacio en un nombre de variable, Python cree que se trata
de dos operandos sin ningún operador:

\beforeverb
\begin{verbatim}
>>> nombre incorrecto = 5
SyntaxError: invalid syntax
\end{verbatim}
\afterverb
%
Para la mayoría de errores de sintaxis, los mensajes de error no ayudan mucho.
Los mensajes más comunes son {\tt SyntaxError: invalid syntax} y
{\tt SyntaxError: invalid token}, ninguno de los cuales resulta muy informativo.

\index{mensaje de error}
\index{use before def}
\index{exception}
\index{runtime error}
\index{error!runtime}

El runtime error (error en tiempo de ejecución) que es más probable que obtengas es un
``use before def'' (uso antes de definir); que significa que estás intentando usar una variable
antes de que le hayas asignado un valor. Eso puede ocurrir si escribes mal el nombre de la variable: 

\beforeverb
\begin{verbatim}
>>> capital = 327.68
>>> interes = capitla * tipo
NameError: name 'capitla' is not defined
\end{verbatim}
\afterverb
%
Los nombres de las variables son sensibles a mayúsculas, así que {\tt LaTeX} no es
lo mismo que {\tt latex}.

\index{sensibilidad a mayúsculas, nombres de variables}
\index{semántico, error}
\index{error!semántico}

En este punto, la causa más probable de un error semántico es
el orden de las operaciones. Por ejemplo, para evaluar $\frac{1}{2 \pi}$,
puedes sentirte tentado a escribir

\beforeverb
\begin{verbatim}
>>> 1.0 / 2.0 * pi
\end{verbatim}
\afterverb
%
Pero la división se evalúa antes, ¡así que obtendrás $\pi / 2$, que
no es lo mismo! No hay forma de que Python
sepa qué es lo que querías escribir exactamente, así que en este caso no
obtienes un mensaje de error; simplemente obtienes una respuesta incorrecta.

\index{orden de operaciones}


\section{Glosario}

\begin{description}

\item[asignación:]  Una sentencia que asigna un valor a una variable.
\index{asignación}

\item[cadena:] Un tipo que representa secuencias de caracteres.
\index{cadena}

\item[concatenar:]  Unir dos operandos, uno a continuación del otro.
\index{concatenación}

\item[comentario:]  Información en un programa que se pone para otros
programadores (o para cualquiera que lea el código fuente), y no tiene efecto alguno
en la ejecución del programa.
\index{comentarios}

\item[división entera:] La operación que divide dos números y trunca la
parte fraccionaria.
\index{división!entera}

\item[entero:] Un tipo que representa números enteros.
\index{entero}

\item[evaluar:]  Simplificar una expresión realizando las operaciones
en orden para obtener un único valor.
\index{evaluar}

\item[expresión:]  Una combinación de variables, operadores y valores que
representan un único valor resultante.
\index{expresión}

\item[mnemónico:] Una ayuda para memorizar. A menudo damos nombres mnemónicos a las variables
para ayudarnos a recordar qué está almacenado en ellas.
\index{mnemónico}

\item[palabra clave:]  Una palabra reservada que es usada por el compilador para analizar un
programa; no se pueden usar palabres clave como {\tt if}, {\tt  def}, y {\tt while} como
nombres de variables.
\index{palabra clave}

\item[punto flotante:] Un tipo que representa números con parte
decimal.
\index{punto-flotante}

\item[operador:]  Un símbolo especial que representa un cálculo simple, como
suma, multiplicación o concatenación de cadenas.
\index{operador}

\item[operador módulo:]  Un operador, representado por un signo de porcentaje
({\tt \%}), que funciona con enteros y obtiene el resto cuando
un número es dividido por otro.
\index{módulo, operador}
\index{operador!módulo}

\item[operando:]  Uno de los valores con los cuales un operador opera.
\index{operando}

\item[reglas de precedencia:]  El conjunto de reglas que gobierna el orden en el cual
son evaluadas las expresiones que involucran a múltiples operadores.
\index{reglas de precedencia}
\index{precedencia}

\item[sentencia:]  Una sección del código que representa un comando o acción.
Hasta ahora, las únicas sentencias que hemos visto son asignaciones y sentencias print.
\index{sentencia}

\item[tipo:] Una categoría de valores. Los tipos que hemos visto hasta ahora
son enteros (tipo {\tt int}), números en punto flotante (tipo {\tt
float}), y cadenas (tipo {\tt str}).
\index{tipo}

\item[valor:] Una de las unidades básicas de datos, como un número o una cadena,
que un programa manipula.
\index{valor}

\item[variable:] Un nombre que hace referencia a un valor.
\index{variable}

\end{description}

\section{Ejercicios}

\begin{ex}
Escribe un programa que use \verb"raw_input" para pedirle al usuario su nombre
y luego darle la bienvenida.

\begin{verbatim}
Introduzca tu nombre: Chuck
Hola, Chuck
\end{verbatim}

\end{ex}

\begin{ex}
Escribe un programa para pedirle al usuario el número de horas y la tarifa por hora para calcular
el salario bruto.
\begin{verbatim}
Introduzca Horas: 35
Introduzca Tarifa: 2.75
Salario: 96.25
\end{verbatim}
\end{ex}
%
Por ahora no es necesario preocuparse de que nuestro salario tenga exactamente dos
dígitos después del punto decimal. Si quieres, puedes probar la función interna de Python
{\tt round} para redondear de forma adecuada el salario resultante
a dos dígitos decimales.

\begin{ex}
Asume que ejecutamos las siguientes sentencias de asignación:

\begin{verbatim}
ancho = 17
alto = 12.0
\end{verbatim}

Para cada una de las expresiones siguientes, escribe el valor de la
expresión y el tipo (del valor de la expresión).

\begin{enumerate}

\item {\tt ancho/2}

\item {\tt ancho/2.0}

\item {\tt alto/3}

\item {\tt 1 + 2 * 5}

\end{enumerate}

Usa el intérprete de Python para comprobar tus respuestas.
\end{ex}

\begin{ex}
Escribe un programa que le pida al usuario una temperatura en grados Celsius,
la convierta a grados Fahrenheit e imprima por pantalla
la temperatura convertida.
\end{ex}


