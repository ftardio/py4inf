% The contents of this file is 
% Copyright (c) 2009-  Charles R. Severance, All Righs Reserved

\chapter{Colaboraciones}
\section*{Lista de colaboradores de ``Python para Informáticos''}

Bruce Shields por la edición de la copia de los primeros borradores,
Sarah Hegge,
Steven Cherry,
Sarah Kathleen Barbarow,
Andrea Parker,
Radaphat Chongthammakun,
Megan Hixon,
Kirby Urner,
Sarah Kathleen Barbrow,
Katie Kujala,
Noah Botimer,
Emily Alinder,
Mark Thompson-Kular,
James Perry,
Eric Hofer,
Eytan Adar,
Peter Robinson,
Deborah J. Nelson,
Jonathan C. Anthony,
Eden Rassette,
Jeannette Schroeder,
Justin Feezell,
Chuanqi Li,
Gerald Gordinier,
Gavin Thomas Strassel,
Ryan Clement,
Alissa Talley,
Caitlin Holman,
Yong-Mi Kim,
Karen Stover,
Cherie Edmonds,
Maria Seiferle,
Romer Kristi D. Aranas (RK),
Grant Boyer,
Hedemarrie Dussan,
Fernando Tardío por la traducción al español.

% CONTRIB

\section*{Prefacio para ``Think Python''}

\subsection*{La extraña historia de ``Think Python''}

(Allen B. Downey)

En Enero de 1999, estaba preparándome para enseñar una clase de introducción
a la programación en Java. Había impartido el curso tres veces y me estaba
frustrando. La tasa de fracaso en la clase era demasiado alta e, incluso
aquellos estudiantes que aprobaban, lo hacían con un nivel general de conocimientos
demasiado bajo.

Me di cuenta de que uno de los problemas eran los libros.
Eran demasiado grandes, con demasiados detalles innecesarios de Java, y
sin suficiente orientación de alto nivel sobre cómo programar. Y todos ellos
sufrían el mismo efecto trampilla: comenzaban siendo muy fáciles,
avanzaban poco a poco, y en algún lugar alrededor del Capítulo 5 el suelo
desaparecía. Los estudiantes recibían demasiado material nuevo demasiado rápido,
y yo tenía que pasar el resto del semestre recogiendo los pedazos.

Dos semanas antes del primer día de clase, decidí escribir mi
propio libro.
Mis objetivos eran:

\begin{itemize}

\item Hacerlo breve. Para los estudiantes es mejor leer 10 páginas
que no tener que leer 50.

\item Ser cuidadoso con el vocabulario. Intenté minimizar la jerga
y definir cada término al usarlo la primera vez.

\item Construir poco a poco. Para evitar las trampillas, tomé los temas
más difíciles y los dividí en una serie de pasos más pequeños.

\item Enfocarlo a la programación, no al lenguaje de programación. Incluí
el subconjunto de Java mínimo imprescindible y excluí el resto.

\end{itemize}

Necesitaba un título, de modo que elegí caprichosamente \emph{How to Think Like
a Computer Scientist} (Cómo pensar como un informático).

Mi primera versión era tosca, pero funcionaba. Los estudiantes la leían,
y comprendían lo suficiente como para que pudiera emplear el tiempo de clase
en tratar los temas difíciles, los temas interesantes y (lo más importante) dejar a los
estudiantes practicar.

Publiqué el libro bajo Licencia de Documentación Libre GNU ({\tt GNU Free Documentation License}),
que permite a los usuarios copiar, modificar y distribuir el libro.

\index{GNU Free Documentation License}
\index{Free Documentation License, GNU}

Lo que sucedió después es la parte divertida. Jeff Elkner, un profesor
de escuela secundaria de Virginia, adoptó mi libro y lo tradujo para
Python. Me envió una copia de su traducción, y tuve la inusual
experiencia de aprender Python leyendo mi propio libro.

Jeff y yo revisamos el libro, incorporamos un caso práctico realizado por
Chriss Meyers, y en 2001 publicamos \emph{How to Think Like
a Computer Scientist: Learning with Python} (Cómo pensar como un informático:
Aprendiendo con Python), también bajo
Licencia de Documentación Libre GNU ({\tt GNU Free Documentation License}).
Publiqué el libro como {\tt Green Tea Press} y comencé a vender
copias en papel a través de Amazon.com y librerías universitarias.
Hay otros libros de {\tt Green Tea Press} disponibles en
\url{greenteapress.com}.

En 2003, comencé a impartir clases en el Olin College y tuve que enseñar
Python por primera vez. El contraste con Java fue notable.
Los estudiantes se tenían que esforzar menos, aprendían más, trabajaban
en proyectos más interesantes, y en general se divertían mucho más.

Durante los últimos cinco años he continuado desarrollando el libro,
corrigiendo errores, mejorando algunos de los ejemplos y
añadiendo material, especialmente ejercicios. En 2008 empecé a trabajar
en una revisión general---al mismo tiempo, se puso en contacto conmigo
un editor de la Cambridge University Press interesado en publicar la
siguiente edición. ¡Qué oportuno!

Espero que disfrutes con este libro, y que te ayude
a aprender a programar y a pensar, al menos un poquito, como
un informático.

\subsection*{Agradecimientos por ``Think Python''}

(Allen B. Downey)

Lo primero y más importante, mi agradecimiento a Jeff Elkner por
haber traducido mi libro de Java a Python, ya que eso fue lo que hizo
comenzar este proyecto y me introdujo en el que se ha convertido
en mi lenguaje de programación favorito.

Quiero dar las gracias también a Chris Meyers, que ha contribuído en varias
secciones de \emph{How to Think Like a Computer Scientist}.

Y agradezco a la {\tt Free Software Foundation} (Fundación de Software Libre) por
haber desarrollado la {\tt GNU Free Documentation License}, que ha ayudado
a que mi colaboración con Jeff y Chris fuera posible.

\index{GNU Free Documentation License}
\index{Free Documentation License, GNU}

Tambien quiero agradecer a los editores de Lulu que trabajaron en
\emph{How to Think Like a Computer Scientist}.

Doy las gracias a todos los estudiantes que trabajaron con las
primeras versiones de este libro y a todos los colaboradores (listados
en un Apéndice) que han enviado correcciones y sugerencias.

Y quiero dar las gracias a mi mujer, Lisa, por su trabajo en este libro, en {\tt Green
Tea Press}, y por todo lo demás, también.

Allen B. Downey \\
Needham MA\\

Allen Downey es un Profesor Asociado de Informática en
el {\tt Franklin W. Olin College of Engineering}.

\section*{Lista de colaboradores de ``Think Python''}

\index{colaboradores}

(Allen B. Downey)

Más de 100 lectores perspicaces y atentos me han enviado
sugerencias y correcciones a lo largo de los últimos años. Su
contribución y entusiasmo por este proyecto han resultado de
gran ayuda.

Para conocer los detalles sobre la naturaleza de cada una de las contribuciones
de estas personas, mira en el texto de ``Think Python''.

Lloyd Hugh Allen,
Yvon Boulianne,
Fred Bremmer,
Jonah Cohen,
Michael Conlon,
Benoit Girard,
Courtney Gleason and Katherine Smith,
Lee Harr,
James Kaylin,
David Kershaw,
Eddie Lam,
Man-Yong Lee,
David Mayo,
Chris McAloon,
Matthew J. Moelter,
Simon Dicon Montford,
John Ouzts,
Kevin Parks,
David Pool,
Michael Schmitt,
Robin Shaw,
Paul Sleigh,
Craig T. Snydal,
Ian Thomas,
Keith Verheyden,
Peter Winstanley,
Chris Wrobel,
Moshe Zadka,
Christoph Zwerschke,
James Mayer,
Hayden McAfee,
Angel Arnal,
Tauhidul Hoque and Lex Berezhny,
Dr. Michele Alzetta,
Andy Mitchell,
Kalin Harvey,
Christopher P. Smith,
David Hutchins,
Gregor Lingl,
Julie Peters,
Florin Oprina,
D.~J.~Webre,
Ken,
Ivo Wever,
Curtis Yanko,
Ben Logan,
Jason Armstrong,
Louis Cordier,
Brian Cain,
Rob Black,
Jean-Philippe Rey at Ecole Centrale Paris,
Jason Mader at George Washington University made a number
Jan Gundtofte-Bruun,
Abel David and Alexis Dinno,
Charles Thayer,
Roger Sperberg,
Sam Bull,
Andrew Cheung,
C. Corey Capel,
Alessandra,
Wim Champagne,
Douglas Wright,
Jared Spindor,
Lin Peiheng,
Ray Hagtvedt,
Torsten H\"{u}bsch,
Inga Petuhhov,
Arne Babenhauserheide,
Mark E. Casida,
Scott Tyler,
Gordon Shephard,
Andrew Turner,
Adam Hobart,
Daryl Hammond and Sarah Zimmerman,
George Sass,
Brian Bingham,
Leah Engelbert-Fenton,
Joe Funke,
Chao-chao Chen,
Jeff Paine,
Lubos Pintes,
Gregg Lind and Abigail Heithoff,
Max Hailperin,
Chotipat Pornavalai,
Stanislaw Antol,
Eric Pashman,
Miguel Azevedo,
Jianhua Liu,
Nick King,
Martin Zuther,
Adam Zimmerman,
Ratnakar Tiwari,
Anurag Goel,
Kelli Kratzer,
Mark Griffiths,
Roydan Ongie,
Patryk Wolowiec,
Mark Chonofsky,
Russell Coleman,
Wei Huang,
Karen Barber,
Nam Nguyen,
St\'{e}phane Morin,
Fernando Tard\'{i}o,
y
Paul Stoop.

