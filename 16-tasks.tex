% The contents of this file is 
% Copyright (c) 2009-  Charles R. Severance, All Righs Reserved

\chapter{Automatización de tareas habituales en tu ordenador}

Hemos estado leyendo datos desde ficheros, redes, servicios y
bases de datos. Python puede moverse también a través de todos los
directorios y carpetas de tus ordenadores y además leer
los ficheros.

En este capítulo, vamos a escribir programas que busquen
por todo el ordenador y
realicen ciertas operaciones sobre cada fichero.
Los archivos están organizados en directorios (también llamados ``carpetas'').
Scripts sencillos en Python
pueden ocuparse de tareas simples que se tengan que repetir
sobre cientos o miles de ficheros
distribuidos a lo largo de un arbol de directorios o incluso por todo el ordenador.

Para movernos a través de todos los directorios y archivos de un árbol usaremos
{\tt os.walk} y un bucle {\tt for}. Es similar a cómo
{\tt open} nos permite usar un bucle para leer el contenido de un archivo,
{\tt socket} nos permite usar un bucle para leer el contenido de una conexión de red, y
{\tt urllib} nos permite abrir un documento web y movernos a través de su contenido. 

\section{Nombres de archivo y rutas}
\label{paths}

\index{file name}
\index{path}
\index{directory}
\index{folder}

Cada programa en ejecución tiene su propio ``directorio actual'', que es su
directorio por defecto para la mayoría de las operaciones. Por ejemplo, cuando abres un archivo
en modo lectura, Python lo busca en el directorio actual.

\index{os module}
\index{module!os}

El módulo {\tt os} proporciona funciones para trabajar con archivos y
directorios ({\tt os} significa ``Sistema Operativo''\footnote{en inglés, ``operating system''
(Nota del trad.)}).
{\tt os.getcwd} devuelve el nombre del directorio actual:

\index{getcwd function}
\index{function!getcwd}

\beforeverb
\begin{verbatim}
>>> import os
>>> cwd = os.getcwd()
>>> print cwd
/Users/csev
\end{verbatim}
\afterverb
%
{\tt cwd} significa {\bf directorio de trabajo actual}\footnote{en inglés, ``current working
directory'' (Nota del trad.)}. El resultado en
este ejemplo es {\tt /Users/csev}, que es el directorio home de un
usuario llamado {\tt csev}.

\index{working directory}
\index{directory!working}

Una cadena como {\tt cwd}, que identifica un fichero, recibe el nombre de ruta.
Una {\bf ruta relativa} comienza en el directorio actual;
una {\bf ruta absoluta} comienza en el directorio superior del
sistema de archivos.

\index{relative path}
\index{path!relative}
\index{absolute path}
\index{path!absolute}

Las rutas que hemos visto hasta ahora son simples nombres de archivo, de modo que
son relativas al directorio actual. Para encontrar la ruta absoluta de
un archivo se puede utilizar {\tt os.path.abspath}:

\beforeverb
\begin{verbatim}
>>> os.path.abspath('memo.txt')
'/Users/csev/memo.txt'
\end{verbatim}
\afterverb
%
{\tt os.path.exists} comprueba
si un fichero o directoriio existe:

\index{exists function}
\index{function!exists}

\beforeverb
\begin{verbatim}
>>> os.path.exists('memo.txt')
True
\end{verbatim}
\afterverb
%
Si existe, {\tt os.path.isdir} comprueba si se trata de un directorio:

\beforeverb
\begin{verbatim}
>>> os.path.isdir('memo.txt')
False
>>> os.path.isdir('musica')
True
\end{verbatim}
\afterverb
%
De forma similar, {\tt os.path.isfile} comprueba si se trata de un fichero.

{\tt os.listdir} devuelve una lista de los archivos (y otros directorios)
existentes en el directorio dado:

\beforeverb
\begin{verbatim}
>>> os.listdir(cwd)
['musica', 'fotos', 'memo.txt']
\end{verbatim}
\afterverb
%


\section{Ejemplo: Limpieza del directorio foto}

Hace algún tiempo, construí un software parecido a Flickr, que
recibía fotos desde mi teléfono móvil y las almacenaba
en mi servidor. Lo escribí antes de que Flickr existiera y he continuado
usándolo después, porque quería mantener las copias originales
de mis imágenes para siempre.

También quería enviar una descripción sencilla, de una línea de texto en el mensaje MMS,
en la línea de título del correo. Almacené esos mensajes
en un fichero de texto en el mismo directorio que el fichero con la imagen.
Se me ocurrió una estructura de directorios basada en el mes, año, día y hora en
que cada foto había sido realizada. Lo siguiente sería un ejemplo del nombre
de una foto y su descripción:

\beforeverb
\begin{verbatim}
./2006/03/24-03-06_2018002.jpg
./2006/03/24-03-06_2018002.txt
\end{verbatim}
\afterverb
%
Después de siete años, tenía un montón de fotos y descripciones. A lo largo de los años,
según iba cambiando de teléfono móvil, a veces mi código para extraer el texto de los mensajes
fallaba y añadía un montón de datos inútiles al servidor en lugar de la descripción.

Quería moverme a través de esos ficheros y averiguar cuáles de los
textos eran realmente descripciones y cuales eran basura, para poder eliminar
los ficheros erróneos. Lo primero que hice fue crear un sencillo inventario de
cuántos archivos de texto tenía en uno de los subdirectorios,
usando el programa siguiente:

\beforeverb
\begin{verbatim}
import os
contador = 0
for (nombredir, dirs, ficheros) in os.walk('.'):
   for nombrefichero in ficheros:
       if nombrefichero.endswith('.txt') :
           contador = contador + 1
print 'Ficheros:', contador

python txtcount.py
Ficheros: 1917
\end{verbatim}
\afterverb
%
El trozo de código que hace eso posible es la librería de Python
{\tt os.walk}. Cuando llamamos a {\tt os.walk} y le damos un directorio
de inicio, ``recorrerá''\footnote{``walk'' significa ``recorrer'' (Nota del trad.)}
todos los directorios y subdirectorios de forma recursiva. La cadena ``.'' le indica
que comience en el directorio actual y se mueva hacia delante.
A medida que va encontrando directorios, obtenemos tres valores en una tupla
en el cuerpo del bucle {\tt for}. El primer valor es el nombre del
directorio actual, el segundo es la lista de subdirectorios dentro
del actual y el tercer valor es la lista de ficheros
que se encuentran en el directorio.

No necesitamos mirar explícitamente dentro de cada uno de los subdirectorios,
porque podemos contar con que {\tt os.walk} visitará cada
uno de ellos al final. Pero sí que tendremos que mirar cada fichero, de
modo que usamos un sencillo bucle {\tt for} para examinar cada uno de los archivos
en el directorio actual. Verificamos cada fichero para comprobar si
termina por ``.txt'', y luego contamos el número de
ficheros en todo el árbol de directorios que terminan con ese
sufijo.

Una vez que tenemos una noción acerca de cuántos archivos terminan por ``.txt'', lo
siguiente es intentar determinar automáticamente
desde Python qué ficheros son incorrectos y cuáles están bien.
De modo que escribimos un programa sencillo para imprimir en pantalla los
nombres de los ficheros y el tamaño de cada uno:

\beforeverb
\begin{verbatim}
import os
from os.path import join
for (nombredir, dirs, ficheros) in os.walk('.'):
   for nombrefichero in ficheros:
       if nombrefichero.endswith('.txt') :
           elfichero = os.path.join(nombredir,nombrefichero)
           print os.path.getsize(elfichero), elfichero
\end{verbatim}
\afterverb
%
Ahora en vez de simplemente contar los ficheros, creamos
un nombre de archivo concatenando el nombre del directorio con
el nombre del archivo, usando {\tt os.path.join}.
Es importarnte usar
{\tt os.path.join} en vez de una simple concatenación de cadenas,
porque en Windows para consturir las rutas de archivos
se utiliza la barra-invertida (\verb"\"), mientras que en Linux
o Apple se usa la barra normal (\verb"/").
{\tt os.path.join} conoce esas diferencias y sabe en qué
sistema se está ejecutando, de modo que realiza la concatenación correcta
dependiendo del sistema. Así el mismo código de Python
puede ejecutarse tanto en Windows como en sistemas de estilo Unix.

Una vez que tenemos el nombre del fichero completo con la ruta
del directorio, usamos la utilidad {\tt os.path.getsize}
para obtener el tamaño e imprimirlo en pantalla, produciendo la
salida siguiente:

\beforeverb
\begin{verbatim}
python txtsize.py
...
18 ./2006/03/24-03-06_2303002.txt
22 ./2006/03/25-03-06_1340001.txt
22 ./2006/03/25-03-06_2034001.txt
...
2565 ./2005/09/28-09-05_1043004.txt
2565 ./2005/09/28-09-05_1141002.txt
...
2578 ./2006/03/27-03-06_1618001.txt
2578 ./2006/03/28-03-06_2109001.txt
2578 ./2006/03/29-03-06_1355001.txt
...
\end{verbatim}
\afterverb
%
Si observamos la salida, nos damos cuenta de que algunos ficheros son demasiado pequeños y
otros muchos son demasiado grandes y tienen siempre el mismo tamaño (2578 y 2565).
Cuando echamos un vistazo manualmente a algunos de esos ficheros grandes,
descubrimos que no son nada más que un montón genérico de HTML idéntico, que llega
desde el correo enviado al sistema por mi teléfono T-Mobile:

\beforeverb
\begin{verbatim}
<html>
        <head>
                <title>T-Mobile</title>
...
\end{verbatim}
\afterverb
%
Ojeando uno de estos fichero, da la impresión de que no hay información aprovechable
en ellos, de modo que seguramente se pueden borrar.

Pero antes de borrarlos, escribiremos un programa que busque los ficheros
que tengan más de una línea de longitud y muestre su contenido.
No nos vamos a molestar en mostrarnos a nosotros mismos aquellos ficheros que tengan
un tamaño exacto de 2578 ó 2565 caracteres, porque ya sabemos que esos no contienen
ninguna información útil.

De modo que escribimos el programa siguiente:

\beforeverb
\begin{verbatim}
import os
from os.path import join
for (nombredir, dirs, ficheros) in os.walk('.'):
   for nombrefichero in ficheros:
       if nombrefichero.endswith('.txt') :
           elfichero = os.path.join(nombredir,nombrefichero)
           tamano = os.path.getsize(elfichero)
           if tamano == 2578 or tamano == 2565:
               continue
           manf = open(elfichero,'r')
           lineas = list()
           for linea in manf:
               lineas.append(linea)
           manf.close()
           if len(lineas) > 1:
                print len(lineas), elfichero
                print lineas[:4]
\end{verbatim}
\afterverb
%
Usamos un {\tt continue} para omitir los ficheros con los dos
``tamaños incorrectos'', a continuación vamos abriendo el resto de los archivos,
pasamos las líneas de cada uno de ellos a una lista de Python
y si el archivo tiene más de una línea imprimimos
en pantalla el número de líneas que contiene y el contenido
de las tres primeras.

Parece que filtrando esos ficheros con los tamaños incorrectos, y asumiendo
que todos los que tienen sólo una línea son correctos, se
consiguen unos datos bastante limpios:

\beforeverb
\begin{verbatim}
python txtcheck.py 
3 ./2004/03/22-03-04_2015.txt
['Little horse rider\r\n', '\r\n', '\r']
2 ./2004/11/30-11-04_1834001.txt
['Testing 123.\n', '\n']
3 ./2007/09/15-09-07_074202_03.txt
['\r\n', '\r\n', 'Sent from my iPhone\r\n']
3 ./2007/09/19-09-07_124857_01.txt
['\r\n', '\r\n', 'Sent from my iPhone\r\n']
3 ./2007/09/20-09-07_115617_01.txt
...
\end{verbatim}
\afterverb
%
Pero existe aún un tipo de ficheros molesto:
hay algunos archivos con tres líneas que se han
colado entre mis datos y que contienen
dos líneas en blanco seguidas por una línea que dice
``Sent from my iPhone''. De modo que haremos el siguiente cambio
al programa para tener en cuenta esos ficheros también:

But there is one more annoying pattern of files: 
there are some three-line files that consist of
two blank lines followed by a line that says
``Sent from my iPhone'' that have slipped 
into my data.   So we make the following change
to the program to deal with these files as well.

\beforeverb
\begin{verbatim}
           lineas = list()
           for linea in manf:
               lineas.append(linea)
           if len(lineas) == 3 and lineas[2].startswith('Sent from my iPhone'):
               continue
           if len(lineas) > 1:
                print len(lineas), elfichero
                print lineas[:4]
\end{verbatim}
\afterverb
%
Simplemente comprobamos si tenemos un fichero con tres líneas, y si la tercera
línea comienza con el texto especificado, lo saltamos.

Ahora, cuando ejecutamos el programa, vemos que sólo quedan cuatro ficheros
multi-línea, y todos ellos parecen bastante razonables:

\beforeverb
\begin{verbatim}
python txtcheck2.py 
3 ./2004/03/22-03-04_2015.txt
['Little horse rider\r\n', '\r\n', '\r']
2 ./2004/11/30-11-04_1834001.txt
['Testing 123.\n', '\n']
2 ./2006/03/17-03-06_1806001.txt
['On the road again...\r\n', '\r\n']
2 ./2006/03/24-03-06_1740001.txt
['On the road again...\r\n', '\r\n']
\end{verbatim}
\afterverb
%
Si miras al diseño global de este programa,
hemos ido refinando sucesivamente qué ficheros aceptamos o rechazamos
y una vez que hemos localizado un patrón ``erróneo'', usamos
{\tt continue} para saltar los ficheros que se ajustan a ese patrón, de modo que podríamos
refinar el código para localizar más patrones incorrectos.

Ahora estamos preparados para eliminar los ficheros, así
que vamos a invertir la lógica y en lugar de imprimir en pantalla
los ficheros correctos que quedan, vamos a imprimir los
``erróneos'' que vamos a eliminar.

\beforeverb
\begin{verbatim}
import os
from os.path import join
for (nombredir, dirs, ficheros) in os.walk('.'):
   for nombrefichero in ficheros:
       if nombrefichero.endswith('.txt') :
           elfichero = os.path.join(nombredir,nombrefichero)
           tamano = os.path.getsize(elfichero)
           if tamano == 2578 or tamano == 2565:
               print 'T-Mobile:',elfichero
               continue
           manf = open(elfichero,'r')
           lineas = list()
           for linea in manf:
               lineas.append(linea)
           manf.close()
           if len(lineas) == 3 and lineas[2].startswith('Sent from my iPhone'):
               print 'iPhone:', elfichero
               continue
\end{verbatim}
\afterverb
%
Ahora podemos ver una lista de ficheros candidatos al
borrado, junto con el motivo por el que van a ser eliminados.
El programa produce la salida siguiente:

\beforeverb
\begin{verbatim}
python txtcheck3.py
...
T-Mobile: ./2006/05/31-05-06_1540001.txt
T-Mobile: ./2006/05/31-05-06_1648001.txt
iPhone: ./2007/09/15-09-07_074202_03.txt
iPhone: ./2007/09/15-09-07_144641_01.txt
iPhone: ./2007/09/19-09-07_124857_01.txt
...
\end{verbatim}
\afterverb
%
Podemos ir revisando estos ficheros para asegurarnos de que no hemos
introducido un error en el programa de forma inadvertida, o de que tal vez
nuestra lógica captura algún fichero que no debería capturar.

Una vez hemos comprobado que esta es la lista de los archivos que queremos eliminar,
realizmos los cambios siguientes en el programa:

\beforeverb
\begin{verbatim}
           if tamano == 2578 or tamano == 2565:
               print 'T-Mobile:',elarchivo
               os.remove(elarchivo)
               continue
...
           if len(lineas) == 3 and lineas[2].startswith('Sent from my iPhone'):
               print 'iPhone:', elarchivo
               os.remove(elarchivo)
               continue
\end{verbatim}
\afterverb
%
En esta versión el programa, primero mostraremos los ficheros erróneos
en pantalla y luego los eliminaremos
usando {\tt os.remove}.

\beforeverb
\begin{verbatim}
python txtdelete.py 
T-Mobile: ./2005/01/02-01-05_1356001.txt
T-Mobile: ./2005/01/02-01-05_1858001.txt
...
\end{verbatim}
\afterverb
%
Si quieres divertirte y ejecutas el programa por segunda vez, no producirá ninguna salida,
ya que los ficheros incorrectos ya no estarán.

Si volvemos a ejecutar {\tt txtcount.py}, podemos ver que se han eliminado
899 ficheros incorrectos:
\beforeverb
\begin{verbatim}
python txtcount.py 
Ficheros: 1018
\end{verbatim}
\afterverb
%
En esta sección, hemos seguido una secuencia en la cual usamos
Python en primer lugar para buscar a través de los directorios y archivos comprobando
patrones. Utilizamos Python para, poco a poco, determinar qué queríamos
hacer para limpiar nuestros directorios. Una vez supimos
qué ficheros eran buenos y cuáles inútiles, utilizamos Python
para eliminar los ficheros y realizar la limpieza.

El problema que necesites resolver puede ser bastante sencillo,
y quizás sólo tengas que comprobar los nombres de los ficheros.
O, tal vez necesites leer cada fichero completo y buscar ciertos patrones en el
interior del mismo. A veces necesitarás
leer todos los ficheros y realizar un cambio en
algunos de ellos. Todo esto resulta bastante
sencillo una vez que comprendes cómo utilizar {\tt os.walk}
y las otras utilidades {\tt os}.

\section{Argumentos de línea de comandos}

\index{arguments}

En capítulos anteriores, teníamos varios programas que usaban
\verb"raw_input" para pedir el nombre de un fichero, y luego leían datos
de ese fichero y los procesaban de este modo:

\beforeverb
\begin{verbatim}
nombre = raw_input('Introduce fichero:')
manejador = open(nombre, 'r')
texto = manejador.read()
...
\end{verbatim}
\afterverb
%
Podemos simplificar este programa un poco si tomamos el nombre del fichero
de la línea de comandos al iniciar Python. Hasta ahora,
simplemente ejecutábamos nuestros programas de Python y respondíamos
a la petición de datos de este modo:

\beforeverb
\begin{verbatim}
python words.py
Introduce fichero: mbox-short.txt
...
\end{verbatim}
\afterverb
%
Podemos colocar cadenas adicionales después del nombre del fichero con el código de Python y acceder
a esos {\bf argumentos de línea de comandos} desde el propio programa Python. Aquí tenemos
un programa sencillo que ilustra la lectura de argumentos desde la línea de comandos:

\beforeverb
\begin{verbatim}
import sys
print 'Cantidad:', len(sys.argv)
print 'Tipo:', type(sys.argv)
for arg in sys.argv:
   print 'Argumento:', arg
\end{verbatim}
\afterverb
%
El contenido de {\tt sys.argv} es una lista de cadenas en la cual la primera
es el nombre del programa Python y las siguientes son los argumentos que se han
escrito en la línea de comandos detrás del nombre del fichero de Python.

Lo siguiente muestra nuestro programa leyendo varios argumentos desde la línea de
comandos:

\beforeverb
\begin{verbatim}
python argtest.py hola aquí
Cantidad: 3
Tipo: <type 'list'>
Argumento: argtest.py
Argumento: hola
Argumento: aquí
\end{verbatim}
\afterverb
%
Hay tres argumentos que se han pasado a nuestro programa, en forma de lista con tres elementos.
El primer elemento de la lista es el nombre del fichero (argtest.py) y los otros son
los dos argumentos de línea de comandos que hemos escrito detrás del nombre del fichero.

Podemos reescribir nuestro programa para leer ficheros, tomando el nombre del fichero
desde un argumento de la línea de comandos, de este modo:

\beforeverb
\begin{verbatim}
import sys

nombre = sys.argv[1]
manejador = open(nombre, 'r')
texto = manejador.read()
print nombre, 'tiene', len(texto), 'bytes'
\end{verbatim}
\afterverb
%
Tomamos el segundo argumento de la línea de comandos y lo usamos como nombre para el fichero
(omitiendo el nombre del programa que está en la anterior entrada {\tt [0]}). Abrimos el fichero
y leemos su contenido así:

\beforeverb
\begin{verbatim}
python argfile.py mbox-short.txt
mbox-short.txt tiene 94626 bytes
\end{verbatim}
\afterverb
%
El uso de argumentos de línea de comandos como entrada puede hacer más sencillo reutilizar tus
programa en Python, especialmente cuando sólo necesitas introducir una o dos cadenas.

\section{Pipes (tuberías)}

\index{shell}
\index{pipe}

La mayoría de los sistemas operativos proporcionan una interfaz de línea de comandos,
también conocida como {\bf shell}. Las shells normalmente proporcionan comandos
para navegar por el sistema de ficheros y ejecutar aplicaciones. Por
ejemplo, en Unix se cambia de directorio con {\tt cd},
se muestra el contenido de un directorio con {\tt ls}, y se ejecuta
un navegador web tecleando (por ejemplo) {\tt firefox}.

\index{ls (Unix command)}
\index{Unix command!ls}

Cualquier programa que se ejecute desde la shell puede ser ejecutado
también desde Python usando una {\bf pipe (tubería)}. Una tubería es un objeto
que representa a un proceso en ejecución.

Por ejemplo, el comando de Unix\footnote{Cuando se usan tuberías para comunicarse
con comandos del sistema operativo como {\tt ls}, es importante
que sepas qué sistema operativo estás utilizando y que sólo abras
tuberías hacia comandos que estén soportados en ese sistema operativo.}
{\tt ls -l} normalmente muestra el
contenido del directorio actual (en formato largo). Se puede
ejecutar {\tt ls} con {\tt os.popen}:

\index{popen function}
\index{function!popen}

\beforeverb
\begin{verbatim}
>>> cmd = 'ls -l'
>>> fp = os.popen(cmd)
\end{verbatim}
\afterverb
%
El argumento de {\tt os.popen} es una cadena que contiene un comando de la shell. El
valor de retorno es un puntero a un fichero que se comporta exactamente igual que un fichero
abierto. Se puede leer la salida del proceso {\tt ls} línea a línea
usando {\tt readline}, u obtener todo de una vez
con {\tt read}:

\index{readline method}
\index{method!readline}
\index{read method}
\index{method!read}

\beforeverb
\begin{verbatim}
>>> res = fp.read()
\end{verbatim}
\afterverb
%
Cuando hayas terminado, debes cerrar la tubería como harías con un fichero:

\index{close method}
\index{method!close}

\beforeverb
\begin{verbatim}
>>> stat = fp.close()
>>> print stat
None
\end{verbatim}
\afterverb
%
El valor de retorno es el estado final del proceso {\tt ls};
{\tt None} significa que ha terminado con normalidad (sin errores).

\section{Glosario}

\begin{description}

\item[argumento de línea de comandos:] Parámetros de la línea de comandos que van detrás del nombre
del fichero Python.

\item[checksum:] Ver también {\bf hashing}.  El término ``checksum'' (suma de comprobación)
viene de la necesidad de verificar si los datos se han alterado al
enviarse a través de la red o al escribirse en un medio de almacenamiento y luego
haber sido leídos de nuevo. Cuando los datos son escritos o enviados, el sistema de envío
realiza una suma de comprobación (checksum) y la envía también. Cuando los
datos se leen o reciben, el sistema de recepción re-calcula la suma de comprobación
de esos datos y lo compara con la cifra recibida. Si ambas sumas de comprobación no coinciden,
se asume que los datos se han alterado durante la transmisión.
\index{checksum}

\item[directorio de trabajo actual:] El directorio actual ``en''
que estás. Puedes cambiar el directorio de trabajo usando el comando
{\tt cd} en la interfaz de línea de comandos de la mayoría de los sistemas.
Cuando abres un fichero en Python usando sólo el nombre del fichero sin información
acerca de la ruta, el fichero debe estar en el directorio de trabajo actual
en el cual estás ejecutando el programa.
\index{directory!current}
\index{directory!working}
\index{directory!cwd}

\item[hashing:] Lectura a través de una cantidad potencialmente grande de datos
para producir una suma de comprobación única para esos datos. Las mejores funciones hash
producen muy pocas ``colisiones''. Las colisiones se producen cuando se envían dos cadenas de datos
distintas a la función de hash y ésta devuelve el mismo hash para ambas.
MD5, SHA1, y SHA256 son ejemplos de funciones hash comunmente utilizadas.
\index{hashing}

\item[pipe (tubería):] Un pipe o tubería es una conexión con un programa en ejecución. Se
puede escribir un programa que envíe datos a otro o reciba datos desde ese otro
mediante una tubería. Una tubería es similar a un
{\bf socket}, excepto que una tubería sólo puede utilizarse
para conectar programas en ejecución dentro del mismo ordenador
(es decir, no se puede usar a través de una red).
\index{pipe}

\item[ruta absoluta:] Una cadena que describe dónde está almacenado
un fichero o directorio, comenzando desde la ``parte superior del árbol de directorios'',
de modo que puede usarse para acceder al fichero o directorio, independientemente
de cual sea el directorio de trabajo actual.
\index{path!absolute}

\item[ruta relativa:] Una cadena que describe dónde se almacena
un fichero o directorio, relativo al directorio de trabajo
actual.
\index{path!relative}

\item[shell:] Una interfaz de línea de comandos de un sistema operativo.
También se la llama ``terminal de programas'' en ciertos sistemas. En esta interfaz
se escriben el comando y parámetros en una línea y se pulsa ``intro''
para ejecutarlo.
\index{shell}

\item[walk (recorrer):] Un término que se usa para describir la idea de visitar
el árbol completo de directorios, subdirectorios, sub-subdirectorios,
hasta que se han visitado todos los directorios. A esto se le llama
``recorrer el árbol de directorios''.
\index{walk}

\end{description}


\section{Ejercicios}

\begin{ex}
\label{checksum}

\index{MP3}

En una colección extensa de archivos MP3 puede haber más de una
copia de la misma canción, almacenadas en distintos directorios o con
nombres de archivo diferentes. El objetivo de este ejercicio es buscar
esos duplicados.

\begin{enumerate}

\item Escribe un programa que recorra un directorio y todos sus
subdirectorios, buscando los archivos que tengan un sufijo determinado (como {\tt .mp3})
y liste las parejas de ficheros que tengan el mismo tamaño.
Pista: Usa un diccionario en el cual la clave sea el tamaño
del fichero obtenido con {\tt os.path.getsize} y el valor sea
el nombre de la ruta concatenado con el nombre del fichero.
Cada vez que encuentres un fichero, verifica si ya tienes otro
con el mismo tamaño. Si es así, has localizado un par de duplicados,
de modo que puedes imprimir el tamaño del archivo y los dos nombres
(el guardado en el diccionario y el del fichero que estás comprobando).

\index{duplicate}
\index{MD5 algorithm}
\index{algorithm!MD5}
\index{checksum}

\item Adapta el programa anterior para buscar ficheros que
tengan contenidos duplicados usando un algoritmo de hashing o
{\bf cheksum (suma de comprobación)}. Por ejemplo,
MD5 (Message-Digest algorithm 5) toma un ``mensaje'' de cualquier
longitud y devuelve una ``suma de comprobación'' de 128 bits. La probabilidad
de que dos ficheros con diferentes contenidos devuelvan la misma
suma de comprobación es muy pequeña.

Puedes leer más acerca de MD5 en \url{es.wikipedia.org/wiki/MD5}. El trozo
de código siguiente abre un fichero, lo lee y calcula
su suma de comprobación:

\beforeverb
\begin{verbatim}
import hashlib 
...
           manf = open(elfichero,'r')
           datos = manf.read()
           manf.close()
           checksum = hashlib.md5(datos).hexdigest()
\end{verbatim}
\afterverb
%
Debes crear un diccionario en el cual la suma de comprobación sea la clave
y el nombre del fichero el valor. Cuando calcules una suma de comprobación
y ésta ya se encuentre como clave dentro del diccionario, habrás localizado
dos ficheros con contenido duplicado, de modo que imprime en pantalla el nombre del fichero
que tienes en el diccionario y el del archivo que acabas de leer. He aquí una salida
de ejemplo de la ejecución del programa en una carpeta con archivos de imágenes:

\beforeverb
\begin{verbatim}
./2004/11/15-11-04_0923001.jpg ./2004/11/15-11-04_1016001.jpg
./2005/06/28-06-05_1500001.jpg ./2005/06/28-06-05_1502001.jpg
./2006/08/11-08-06_205948_01.jpg ./2006/08/12-08-06_155318_02.jpg
\end{verbatim}
\afterverb
%
Aparentemente, a veces envío la misma foto más de una vez
o hago una copia de una foto de vez en cuando sin eliminar después
la original.

\end{enumerate}

\end{ex}

