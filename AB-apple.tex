% The contents of this file is 
% Copyright (c) 2009-  Charles R. Severance, All Righs Reserved

\chapter{Programando con Python en Macintosh}

En este apéndice, mostraremos una serie de pasos
para que puedas ejecutar Python en Macintosh. Dado que Python
ya viene incluido en el sistema Operativo Macintosh, sólo
tenemos que aprender cómo editar ficheros y ejecutar programas
de Python en la ventana del terminal.

Existen muchos métodos diferentes que se pueden seguir para editar y ejecutar
programas de Python, y éste es sólo uno que creo que
resulta muy sencillo.

Lo primero que necesitas es instalar un editor de código. No
querrás utilizar TextEdit o Microsoft Word para editar
programas en Python. Los programas deben estar en ficheros de ``texto plano'',
de modo que necesitas un editor que sea capaz de
editar archivos de texto.

Nuestra recomendación como editor para Macintosh es TextWrangler, que
puede descargarse e instalarse desde:

\url{http://www.barebones.com/products/TextWrangler/}

Para crear un programa Python, ejecuta
{\bf TextWrangler} desde tu carpeta {\bf Aplicaciones}.

Vamos a hacer que nuestro primer programa en Python sea:

\beforeverb
\begin{verbatim}
print 'Hola, Chuck'
\end{verbatim}
\afterverb
%
Excepto que deberías cambiarlo para que escriba tu nombre.
Guarda el fichero en una carpeta en tu escritorio llamada
{\tt p4inf}. Es mejor usar nombres de carpeta cortos
y no utilizar espacios en los nombres de carpetas ni de archivos.
Una vez hayas creado la carpeta, guarda el fichero
en {\tt Desktop{\textbackslash}py4inf{\textbackslash}prog1.py}.

Luego ejecuta el programa {\bf Terminal}. El modo más sencillo es
pulsar el icono Spotlight (la lupa) en la esquina superior
derecha de tu pantalla, introducir ``terminal'', y lanzar la
aplicación que aparece.

Siempre empiezas en tu ``directorio home''. Puedes ver el directorio
actual tecleando el comando {\tt pwd} en la ventana del terminal.

\beforeverb
\begin{verbatim}
67-194-80-15:~ csev$ pwd
/Users/csev
67-194-80-15:~ csev$ 
\end{verbatim}
\afterverb
%
tienes que estar en la carpeta que contiene tu programa en Python
para poder ejecutarlo. Usa el comando {\tt cd} para moverte a una nueva carpeta
y luego usa el comando {\tt ls} para mostrar un listado de los ficheros
de esa carpeta.

\beforeverb
\begin{verbatim}
67-194-80-15:~ csev$ cd Desktop
67-194-80-15:Desktop csev$ cd py4inf
67-194-80-15:py4inf csev$ ls
prog1.py
67-194-80-15:py4inf csev$ 
\end{verbatim}
\afterverb
%
Para ejecutar tu programa, simplemente teclea el comando {\tt python} seguido
por el nombre de tu fichero en el indicador de comandos y pulsa intro.

\beforeverb
\begin{verbatim}
67-194-80-15:py4inf csev$ python prog1.py
Hello Chuck
67-194-80-15:py4inf csev$ 
\end{verbatim}
\afterverb
%
Puedes editar el fichero en TextWrangler, guardarlo, y luego volver
a la línea de comandos y ejecutarlo otra vez, tecleando de nuevo
el nombre del fichero en el indicador.

Si te has perdido en la ventana de la línea de comandos, tan solo tienes
que cerrarla y abrir una nueva.

Pista: También puedes pulsar la tecla ``flecha arriba'' en la línea de comandos para
desplazarte hacia atrás y ejecutar de nuevo un comando introducido anteriormente.

Además, deberías buscar en las preferencias de TextWrangler, y ajustarlas para
que sustituya los caracteres de tabulación por cuatro espacios. Esto te ahorrará
un montón de esfuerzo a la hora de localizar errores de justificación en el código.

Puedes encontrar más información sobre la edición y ejecución de
programas en Python en \url{www.py4inf.com}.


