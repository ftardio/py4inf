% LaTeX source for ``Python for Informatics: Exploring Information''
% Copyright (c)  2010-  Charles R. Severance, All Rights Reserved

\chapter{Cadenas}
\label{strings}


\section{Una cadena es una secuencia}
\index{secuencia}
\index{carácter}
\index{corchete, operador}
\index{operador!corchete}

Una cadena es una {\bf secuencia} de caracteres.
Puedes acceder a los caracteres de uno en uno con el
operador corchete:

\beforeverb
\begin{verbatim}
>>> fruta = 'banana'
>>> letra = fruta[1]
\end{verbatim}
\afterverb
%
\index{índice}
La segunda sentencia extrae el carácter cuyo índice de posición es 1 en la
variable {\tt fruta} y lo asigna a la variable {\tt letra}.

La expresión entre corchetes recibe el nombre de {\bf índice}.
El índice indica a qué carácter de la secuencia se
desea acceder (de ahí su nombre).

Pero puede que no obtengas exactamente lo que esperabas:

\beforeverb
\begin{verbatim}
>>> print letra
a
\end{verbatim}
\afterverb
%
Para la mayoría de la gente, la primera letra de \verb"'banana'" es {\tt b}, no
{\tt a}. Pero en \mbox{Python}, el índice es un seguimiento desde el
comienzo de la cadena, y el seguimiento para la primera letra es cero.

\beforeverb
\begin{verbatim}
>>> letra = fruta[0]
>>> print letra
b
\end{verbatim}
\afterverb
%
De modo que {\tt b} es la ``cero-ésima'' letra de \verb"'banana'", {\tt a}
es la primera letra, y {\tt n} es la segunda.

\beforefig
\centerline{\includegraphics[height=0.50in]{figs2/string.eps}}
\afterfig

\index{índice!comienza en cero}
\index{cero, índice comienza en}

Se puede utilizar cualquier expresión, incluyendo variables y operadores, como índice,
pero el valor del índice debe ser un entero. Si no es así
obtendrás:

\index{índice}
\index{exception!TypeError}
\index{TypeError}

\beforeverb
\begin{verbatim}
>>> letra = fruta[1.5]
TypeError: string indices must be integers
\end{verbatim}
\afterverb
%

\section{Obtener la longitud de una cadena mediante {\tt len}}

\index{len, función}
\index{función!len}

{\tt len} es una función interna que devuelve el número de caracteres
de una cadena:

\beforeverb
\begin{verbatim}
>>> fruta = 'banana'
>>> len(fruta)
6
\end{verbatim}
\afterverb
%
Para obtener la última letra de una cadena, puedes sentirte tentado a intentar
algo como esto:

\index{exception!IndexError}
\index{IndexError}

\beforeverb
\begin{verbatim}
>>> longitud = len(fruta)
>>> ultima = fruta[longitud]
IndexError: string index out of range
\end{verbatim}
\afterverb
%
La razón de que se produzca un {\tt IndexError} es que no hay ninguna letra
en {\tt 'banana'} cuyo índice sea 6. Dado que comenzamos a contar desde cero, las
seis letras son numeradas desde 0 hasta 5. Para obtener el último carácter, debes
restar 1 a {\tt length}:

\beforeverb
\begin{verbatim}
>>> ultima = fruta[longitud-1]
>>> print ultima
a
\end{verbatim}
\afterverb
%
Alternativamente, se pueden usar índices negativos, que cuentan hacia atrás desde
el final de la cadena. La expresión {\tt fruta[-1]} recoge la última
letra, {\tt fruta[-2]} extrae la segunda desde el final, y así todas las demás.

\index{índice!negativo}
\index{negativo, índice}


\section{Recorrido a través de una cadena con un bucle}
\label{for}

\index{recorrido}
\index{bucle!recorrido}
\index{for, bucle}
\index{bucle!for}
\index{sentencia!for}
\index{recorrido}

Gran parte de los cálculos implican procesar una cadena carácter por
carácter. A menudo se empieza por el principio, se van seleccionando caracteres
de uno en uno, se hace algo con ellos, y se continúa hasta el final. Este modelo
de procesado recibe el nombre de {\bf recorrido}. Una forma de escribir un recorrido
es usar un bucle {\tt while}:

\beforeverb
\begin{verbatim}
indice = 0
while indice < len(fruta):
    letra = fruta[indice]
    print letra
    indice = indice + 1
\end{verbatim}
\afterverb
%
Este bucle recorre la cadena y muestra cada letra en su propia línea.
La condición del bucle es {\tt indice < len(fruta)}, de modo
que cuando {\tt indice} es igual a la longitud de la cadena, la
condición es falsa, y el cuerpo del bucle no se ejecuta. El
último carácter al que se accede es el que tiene el índice {\tt len(fruta)-1},
que resulta ser el último carácter de la cadena.

\begin{ex}
Escribe un bucle {\tt while} que comience en el último carácter de la cadena
y haga su recorrido hacia atrás hasta el primer carácter de la misma,
mostrando cada letra en una línea separada.
\end{ex}

Otro modo de escribir un recorrido es con un bucle {\tt for}:

\beforeverb
\begin{verbatim}
for car in fruta:
    print car
\end{verbatim}
\afterverb
%
Cada vez que se recorre el bucle, el carácter siguiente de la cadena es asignado
a la variable {\tt car}. El bucle continúa hasta que no quedan caracteres.


\section{Rebanado de cadenas}
\label{slice}

\index{rebanada, operador}
\index{operador!rebanada}
\index{índice!rebanada}
\index{cadena!rebanada}
\index{rebanada!cadena}

Un segmento de una cadena recibe el nombre de {\bf rebanada (slice)}.
Seleccionar una rebanada es similar a seleccionar caracteres:

\beforeverb
\begin{verbatim}
>>> s = 'Monty Python'
>>> print s[0:5]
Monty
>>> print s[6:12]
Python
\end{verbatim}
\afterverb
%
El operador {\tt [n:m]} devuelve la parte de la cadena desde el
``n-ésimo'' carácter hasta el ``m-ésimo'', incluyendo el primero
pero excluyendo el último.

Si se omite el primer índice (el que va antes de los dos-puntos), la rebanada comenzará
al principio de la cadena. Si el que se omite es el segundo, la rebanada
abarcará hasta el final de la cadena: 

\beforeverb
\begin{verbatim}
>>> fruta = 'banana'
>>> fruta[:3]
'ban'
>>> fruta[3:]
'ana'
\end{verbatim}
\afterverb
%
Si el primer índice es mayor o igual que el segundo, el resultado será
una {\bf cadena vacía}, representada por dos comillas:

\index{comillas}

\beforeverb
\begin{verbatim}
>>> fruta = 'banana'
>>> fruta[3:3]
''
\end{verbatim}
\afterverb
%
Una cadena vacía no contiene caracteres y tiene una longitud 0, pero por
lo demás es exactamente igual que cualquier otra cadena.

\begin{ex}
Dado que {\tt fruta} es una cadena, ¿qué significa
{\tt fruta[:]}?

\index{copiar!rebanada}
\index{rebanada!copiar}


\end{ex}


\section{Las cadenas son inmutables}
\index{mutabilidad}
\index{inmutabilidad}
\index{cadena!inmutable}

Resulta tentador el utilizar el operador {\tt []} en la parte izquierda de una
asignación, con la intención de cambiar un carácter en una cadena.
Por ejemplo:

\index{TypeError}
\index{exception!TypeError}

\beforeverb
\begin{verbatim}
>>> saludo = '¡Hola, mundo!'
>>> saludo[0] = 'J'
TypeError: object does not support item assignment
\end{verbatim}
\afterverb
%
Error de tipado: El objeto no soporta la asignación del elemento.
El ``object'' (objeto) en este caso es la cadena y el ``item'' (elemento) es
el carácter que intentabas asignar. Por ahora, consideraremos que un {\bf objeto} es
lo mismo que un valor, aunque mejoraremos esa definición más adelante.
Un {\bf elemento} es uno de los valores en una secuencia.

\index{objecto}
\index{elemento, asignación}
\index{asignación!elemento}
\index{inmutabilidad}

La razón del error es que las
cadenas son {\bf inmutables}, lo cual significa que no se puede
cambiar una cadena existente. Lo mejor que se puede hacer en estos casos es
crear una cadena nueva que sea una variación de la original:

\beforeverb
\begin{verbatim}
>>> saludo = '¡Hola, mundo!'
>>> nuevo_saludo = 'J' + saludo[1:]
>>> print nuevo_saludo
JHola, mundo!
\end{verbatim}
\afterverb
%
Este ejemplo concatena una primera letra nueva en
una rebanada de {\tt saludo}. Esto conserva intacta
la cadena original.

\index{concatenación}

\section{Bucles y recuentos}
\label{counter}

\index{contador}
\index{contador y bucle}
\index{bucle y contador}
\index{bucle!con cadenas}

El siguiente programa cuenta el número de veces que aparece la letra
{\tt a} en una cadena:

\beforeverb
\begin{verbatim}
palabra = 'banana'
contador = 0
for letra in palabra:
    if letra == 'a':
        contador = contador + 1
print contador
\end{verbatim}
\afterverb
%
Este programa demuestra otro patrón del cálculo llamado {\bf contador}.
La variable {\tt contador} es inicializada a 0 y después
es incrementada cada vez que se encuentra una {\tt a}.
Cuando se sale del bucle, {\tt contador}
contiene el resultado---el número total de {\tt a}'es

\begin{ex}
\index{encapsulación}

Encapsula el código anterior en una función llamada
{\tt contador}, y generalízala, de modo que acepte la cadena y la
letra como argumentos.
\end{ex}

\section{El operador {\tt in}}
\label{inboth}

\index{in, operador}
\index{operador!in}
\index{booleano, operador}
\index{operador!booleano}

La palabra {\tt in} es un operador booleano que toma dos cadenas y
devuelve {\tt True (verdadero)} si la primera aparece como subcadena
dentro de la segunda:

\beforeverb
\begin{verbatim}
>>> 'a' in 'banana'
True
>>> 'seed' in 'banana'
False
\end{verbatim}
\afterverb
%

\section{Comparación de cadenas}

\index{cadena!comparación}
\index{comparación!cadena}

El operador de comparación funciona con cadenas. Para comprobar si dos cadenas son iguales:

\beforeverb
\begin{verbatim}
if palabra == 'banana':
    print  'De acuerdo, bananas.'
\end{verbatim}
\afterverb
%
Otros operadores de comparación resultan útiles para colocar las palabras en orden
alfabético:

\beforeverb
\begin{verbatim}
if palabra < 'banana':
    print 'Tu palabra,' + palabra + ', va antes que banana.'
elif palabra > 'banana':
    print 'Tu palabra,' + palabra + ', va después que banana.'
else:
    print 'De acuerdo, bananas.'
\end{verbatim}
\afterverb
%
Python no maneja las mayúsculas y minúsculas del mismo modo en que
lo hacen las personas. Todas las letras mayúsculas van antes que las
minúsculas, de modo que:

\beforeverb
\begin{verbatim}
Tu palabra, Piña, va antes que banana.
\end{verbatim}
\afterverb
%
Un método habitual para evitar este problema es convertir las cadenas
a un formato estándar, por ejemplo todas en minúsculas, antes de realizar la
comparación. Tenlo en cuenta en el caso de que tengas que defenderte
de un hombre armado con una Piña.


\section{Métodos de {\tt cadenas}}

Las cadenas son un ejemplo de {\bf objetos} en Python. Un objeto contiene
tanto datos (la propia cadena en si misma) como {\bf métodos}, que
en realidad son funciones que están construidas dentro de los propios objetos y
que están disponibles para cualquier {\bf instancia} del objeto.

Python dispone de una función llamada {\tt dir} que lista los métodos disponibles
para un objeto. La función {\tt type} muestra el tipo de cualquier objeto
y la función {\tt dir} muestra los métodos disponibles. 

\beforeverb
\begin{verbatim}
>>> cosa = '¡Hola, mundo!'
>>> type(cosa)
<type 'str'>
>>> dir(cosa)
['capitalize', 'center', 'count', 'decode', 'encode', 
'endswith', 'expandtabs', 'find', 'format', 'index', 
'isalnum', 'isalpha', 'isdigit', 'islower', 'isspace', 
'istitle', 'isupper', 'join', 'ljust', 'lower', 'lstrip', 
'partition', 'replace', 'rfind', 'rindex', 'rjust', 
'rpartition', 'rsplit', 'rstrip', 'split', 'splitlines', 
'startswith', 'strip', 'swapcase', 'title', 'translate', 
'upper', 'zfill']
>>> help(str.capitalize)
Help on method_descriptor:

capitalize(...)
    S.capitalize() -> string
    
    Return a copy of the string S with only its first character
    capitalized.
>>>
\end{verbatim}
\afterverb
%

A pesar de que la función {\tt dir} lista los métodos, y de que
puedes usar {\tt help} para obtener un poco de información sobre
cada método, una fuente de documentación mejor para los métodos de las cadenas
se puede encontrar en
\url{https://docs.python.org/2/library/stdtypes.html#string-methods}.

Llamar a un {\bf método} es similar a llamar a una función---toma
argumentos y devuelve un valor---pero la sintaxis es diferente.
Un método se usa uniendo el nombre del método al de la variable,
utilizando el punto como delimitador.

Por ejemplo, el
método {\tt upper} toma una cadena y devuelve otra nueva con todas
las letras en mayúsculas:

\index{método}
\index{cadena!método}

En vez de usar la sintaxis de función {\tt upper(palabra)}, se usa
la sintaxis de método {\tt palabra.upper()}.

\index{punto, notación}

\beforeverb
\begin{verbatim}
>>> palabra = 'banana'
>>> nueva_palabra = palabra.upper()
>>> print nueva_palabra
BANANA
\end{verbatim}
\afterverb
%
Esta forma de notación con punto especifica el nombre del método,
{\tt upper}, y el nombre de la cadena a la cual se debe aplicar ese método,
{\tt palabra}. Los paréntesis vacíos indican que el método no toma
argumentos.

\index{paréntesis!vacíos}

Una llamada a un método se denomina {\bf invocación}; en este caso, diríamos
que estamos invocando el método {\tt upper} de {\tt palabra}.

\index{invocación}

Por ejemplo, he aquí un método de cadena llamado {\tt find}, que
busca la posición de una cadena dentro de otra:

\beforeverb
\begin{verbatim}
>>> palabra = 'banana'
>>> indice = palabra.find('a')
>>> print indice
1
\end{verbatim}
\afterverb
%
En este ejemplo, se invoca el método {\tt find} de {\tt palabra} y se le
pasa como parámetro la letra que estamos buscando.

El método {\tt find} puede encontrar tanto subcadenas como caracteres:

\beforeverb
\begin{verbatim}
>>> palabra.find('na')
2
\end{verbatim}
\afterverb
%
Puede tomar un segundo argumento que indica en qué posición debe comenzar la búsqueda:

\index{opcional, argumento}
\index{argumento!opcional}

\beforeverb
\begin{verbatim}
>>> palabra.find('na', 3)
4
\end{verbatim}
\afterverb
%
Una tarea habitual es eliminar espacios en blanco (espacios, tabulaciones, saltos de línea)
del comienzo y del final de una cadena usando el método {\tt strip}:

\beforeverb
\begin{verbatim}
>>> linea = '  Y allá vamos  '
>>> linea.strip()
'Y allá vamos'
\end{verbatim}
\afterverb
%
Algunos métodos como {\bf startswith} devuelven valores booleanos.

\beforeverb
\begin{verbatim}
>>> linea = 'Que tengas un buen día'
>>> linea.startswith('Que')
True
>>> linea.startswith('q')
False
\end{verbatim}
\afterverb
%
Te habrás fijado que {\tt startswith} necesita que las mayúsculas también coincidan, de modo
que a veces tomaremos una línea y la convertiremos por completo a minúsculas antes de hacer
ninguna comprobación, usando para ello el método {\tt lower}.

\beforeverb
\begin{verbatim}
>>> linea = 'Que tengas un buen día'
>>> linea.startswith('q')
False
>>> linea.lower()
'que tengas un buen día'
>>> linea.lower().startswith('q')
True
\end{verbatim}
\afterverb
%
En el último ejemplo, se llama al método {\tt lower}
y después se usa {\tt startswith} para comprobar
si la cadena resultante en minúsculas
comienza por la letra ``q''. Mientras tengamos cuidado
con el orden en que las aplicamos, podemos hacer múltiples
llamadas a métodos en una única expresión..

\begin{ex}
\index{count, método}
\index{método!count}

Existe un método de cadena llamado {\tt count}, que es similar a la función
que vimos en el ejercicio anterior. Lee la documentación
de este método en
\url{https://docs.python.org/2/library/stdtypes.html#string-methods}
y escribe una invocación que cuente el número de veces que
aparece la letra ``a''
en \verb"'banana'".
\end{ex}

\section{Análisis de cadenas}

A menudo tendremos que mirar en el interior una cadena para localizar una subcadena. Por
ejemplo, si se nos presentan una serie de líneas formateadas de este modo:

\beforeverb
\begin{alltt}
From stephen.marquard@{\bf uct.ac.za} Sat Jan  5 09:14:16 2008
\end{alltt}
\afterverb

y queremos extraer sólo la segunda mitad de la dirección (es decir,
{\tt uct.ac.za}) de cada línea, podemos hacerlo usando el método
{\tt find} y rebanando la cadena.

En primer lugar, buscaremos la posición del símbolo arroba en la cadena. Después,
buscaremos la posición del primer espacio \emph{después} de la arroba. Y a continuación
rebanaremos la cadena para extraer la porción de la misma que estamos
buscando.

\beforeverb
\begin{verbatim}
>>> datos = 'From stephen.marquard@uct.ac.za Sat Jan  5 09:14:16 2008'
>>> pos_arroba = datos.find('@')
>>> print pos_arroba
21
>>> pos_esp = datos.find(' ',pos_arroba)
>>> print pos_esp
31
>>> host = data[pos_arroba+1:pos_esp]
>>> print host
uct.ac.za
>>> 
\end{verbatim}
\afterverb
%
Usamos la versión del método {\tt find} que nos permite especificar
una posición en la cadena desde la cual queremos que {\tt find} empiece a buscar.
Cuando rebanamos, extraemos los caracteres
desde ``uno más allá de la arroba hasta (\emph{pero no incluyendo}) el
carácter espacio''.

La documentación para el método {\tt find} está disponible en 
\url{https://docs.python.org/2/library/stdtypes.html#string-methods}.

\section{Operador de formato}

\index{formato, operador}
\index{operador!formato}

El {\bf operador de formato} {\tt \%},
nos permite construir cadenas, reemplazando parte de esas cadenas
con los datos almacenados en variables.
Cuando se aplica a enteros, {\tt \%} es el operador módulo. Pero
cuando el primer operando es una cadena, {\tt \%} es el operador formato.

\index{formateo de cadenas}

El primer operando es la {\bf cadena a formatear}, que contiene
una o más {\bf secuencias de formato}, que especifican cómo
será formateado el segundo operador. El resultado es una cadena.

\index{formateo de secuencias}

Por ejemplo, la secuencia de formato \verb"'%d'" quiere decir
que el segundo operador debe ser formateado como un
entero ({\tt d} indica ``decimal''):

\beforeverb
\begin{verbatim}
>>> camellos = 42
>>> '%d' % camellos
'42'
\end{verbatim}
\afterverb
%
El resultado es la cadena \verb"'42'", que no hay que confundir
con el valor entero {\tt 42}.

Una secuencia de formato puede aparecer en cualquier sitio de la cadena,
de modo que puedes insertar un valor en una frase:

\beforeverb
\begin{verbatim}
>>> camellos = 42
>>> 'He visto %d camellos.' % camellos
'He visto 42 camellos.'
\end{verbatim}
\afterverb
%
Si hay más de una secuencia de formato en la cadena,
el segundo argumento debe ser una tupla\footnote{Una tupla es una
secuencia de valores separados por comas dentro de unos paréntesis.
Veremos las tuplas en el capítulo 10}. Cada secuencia de formato se
corresponde con un elemento de la tupla, en orden.

El ejemplo siguiente usa \verb"'%d'" para formatear un entero,
\verb"'%g'" para formatear
un número en punto flotante (no preguntes por qué), y \verb"'%s'" para
formatear una cadena:

\beforeverb
\begin{verbatim}
>>> 'En %d años he visto %g %s.' % (3, 0.1, 'camellos')
'En 3 años he visto 0.1 camellos.'
\end{verbatim}
\afterverb
%
El número de elementos en la tupla debe coincidir con el número
de secuencias de formato en la cadena. El tipo de los
elementos debe coincidir también con las secuencias de formato:

\index{exception!TypeError}
\index{TypeError}

\beforeverb
\begin{verbatim}
>>> '%d %d %d' % (1, 2)
TypeError: not enough arguments for format string
>>> '%d' % 'dólares'
TypeError: illegal argument type for built-in operation
\end{verbatim}
\afterverb
%
En el primer ejemplo, no hay suficientes elementos; en el
segundo, el elemento es de tipo incorrecto.

El operador de formato es potente, pero puede resultar difícil de utilizar.
Puedes leer más sobre él en
\url{https://docs.python.org/2/library/stdtypes.html#string-formatting}.

% Se puede especificar el número de dígitos como parte de la secuencia de formato.
% Por ejemplo, la secuencia \verb"'%8.2f'"
% formatea un número en punto flotante para que tenga un longitud de 8 caracteres, con
% 2 dígitos después del punto decimal:

% \beforeverb
% \begin{verbatim}
% >>> '%8.2f' % 3.14159
% '    3.14'
% \end{verbatim}
% \afterverb
% %
% El resultado ocupa ocho espacios, con dos
% dígitos detrás del punto decimal.  


\section{Depuración}
\index{depuración}

Una habilidad que deberás desarrollar cuando programes es la de
estar preguntándote siempre: ``¿Qué podría ir mal aquí?'', o quizás,
``¿Qué locura puede hacer el usuario para destrozar nuestro (aparentemente)
perfecto programa?''

Por ejemplo, mira el programa que usamos para demostrar el bucle
{\tt while} en el capítulo dedicado a la iteración:

\beforeverb
\begin{verbatim}
while True:
    linea = raw_input('> ')
    if linea[0] == '#' :
        continue
    if linea == 'fin':
        break
    print linea

print '¡Terminado!'
\end{verbatim}
\afterverb
%
Mira lo que sucede cuando el usuario introduce una línea vacía como entrada:

\beforeverb
\begin{verbatim}
> hola a todos
hola a todos
> # no imprimas esto
> ¡imprime esto!
¡imprime esto!
> 
Traceback (most recent call last):
  File "copytildone.py", line 3, in <module>
    if linea[0] == '#' :
\end{verbatim}
\afterverb
%
El código funciona hasta que se le presenta una línea vacía. Entonces,
como no hay carácter cero-ésimo, obtenemos un traceback. Existen dos
soluciones a esto para convertir la línea tres en ``segura'', incluso
cuando la entrada sea una cadena vacía.

Una posibilidad es simplemente usar el método {\tt startswith},
que devuelve {\tt False (falso)} si la cadena está vacía.

\beforeverb
\begin{verbatim}
    if linea.startswith('#') :
\end{verbatim}
\afterverb
%
\index{guardián, patrón}
\index{patrón!guardián}
Otro modo es asegurar la sentencia {\tt if} usando el patrón {\bf guardián},
y asegurarnos de que la segunda expresión lógica sea evaluada
sólo cuando hay al menos un carácter en la cadena:

\beforeverb
\begin{verbatim}
    if len(linea) > 0 and linea[0] == '#' :
\end{verbatim}
\afterverb
%

\section{Glosario}

\begin{description}

\item[búsqueda:] Un patrón de recorrido que se detiene
cuando encuentra lo que está buscando.
\index{búsqueda, patrón}
\index{patrón!de búsqueda}

\item[cadena vacía:] Una cadena sin caracteres y de longitud 0, representada por
dos comillas.
\index{cadena vacía}

\item[contador:] Una variable utilizada para contar algo, normalmente inicializada
a cero y luego incrementada.
\index{contador}

\item[cadena a formatear:] Una cadena, usada con el operador de formato,
que contiene secuencias de formato.
\index{cadena a formatear}

\item[elemento:] Uno de los valores en una secuencia.
\index{elemento}

\item[flag (bandera):] Una variable booleana que se usa para indicar si
una condición es verdadera.
\index{flag}
\index{bandera}

\item[índice:] Un valor entero usado para seleccionar un elemento en
una secuencia, como un carácter en una cadena.
\index{índice}

\item[inmutable:] La propiedad de una secuencia cuyos elementos
no pueden ser asignados.
\index{inmutabilidad}

\item[invocación:] Una sentencia que llama a un método.
\index{invocación}

\item[método:] Una función que está asociada con un objeto y es llamada
usando la notación punto.
\index{método}

\item[objecto:] Algo a lo que puede referirse una variable. Por ahora,
puedes usar ``objeto'' y ``valor'' indistintamente.
\index{objecto}

\item[operador de formato:] Un operador, {\tt \%}, que toma una cadena 
a formatear y una tupla y genera una cadena que incluye
los elementos de la tupla formateados como se especifica en la cadena de formato.
\index{formato, operador}
\index{operador!formato}

\item[rebanada (slice):] Una parte de una cadena especificada por un rango de índices.
\index{rebanada}
\index{slice}

\item[recorrido:] Iterar a través de los elementos de una secuencia,
realizando un operación similar en cada uno de ellos.
\index{recorrido}

\item[secuencia:] Un conjunto ordenado; es decir, un conjunto de
valores donde cada valor está identificado por un índice entero.
\index{secuencia}

\item[secuencia de formato:] Una secuencia de caracteres en una cadena de formato,
como {\tt \%d}, que especifica cómo debe ser formateado un valor.
\index{formato, secuencia de}

\end{description}


\section{Ejercicios}

\begin{ex}
Toma el código en Python siguiente, que almacena una cadena:`

\beforeverb
\begin{alltt}
cad = 'X-DSPAM-Confidence: {\bf 0.8475}'
\end{alltt}
\afterverb

Usa {\tt find} y rebanado de cadenas para extraer la porción
de la cadena después del carácter punto, y luego usa la
función {\tt float} para convertir la cadena extraída en
un número en punto flotante.

\end{ex}


\begin{ex}
\index{cadena, métodos}
\index{métodos!cadena}

Lee la documentación de los métodos de cadena que está en
\url{https://docs.python.org/2/library/stdtypes.html#string-methods}.
Puede que quieras experimentar con algunos de ellos para asegurarte
de que comprendes cómo funcionan. {\tt strip} y
{\tt replace} resultan particularmente útiles.

La documentación utiliza una sintaxis que puede resultar confusa.
Por ejemplo, en \verb"find(sub[, start[, end]])", los corchetes
indican argumentos opcionales. De modo que {\tt sub} es necesario,
pero {\tt start} es opcional, y si incluyes {\tt start},
entonces {\tt end} es opcional.

\end{ex}