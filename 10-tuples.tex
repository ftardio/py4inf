% LaTeX source for ``Python for Informatics: Exploring Information''
% Copyright (c)  2010-  Charles R. Severance, All Rights Reserved

\chapter{Tuplas}
\label{tuplechap}

\section{Las tuplas son inmutables}

\index{tupla}
\index{tipo!tupla}
\index{secuencia}

Una tupla\footnote{Anécdota: La palabra ``tupla (tuple en inglés)'' proviene de los nombres
dados a las secuencias de números de distintas longitudes: simple,
doble, triple, cuádrupe, quíntuple, séxtuple, séptuple, etc.}
es una secuencia de valores muy parecida a una lista.
Los valores almacenados en una tupla pueden ser de cualquier tipo, y
están indexados por enteros.
La diferencia más importante es que las tuplas son {\bf inmutables}.
Las tuplas además son {\bf comparables} y {\bf hashables} (dispersables), de modo que
las listas de tuplas se pueden ordenar y también es posible usar tuplas
como valores para las claves en los diccionarios de Python.

\index{mutabilidad}
\index{hashable}
\index{dispersable}
\index{comparable}
\index{inmutabilidad}

Sintácticamente, una tupla es una lista de valores separados por comas:

\beforeverb
\begin{verbatim}
>>> t = 'a', 'b', 'c', 'd', 'e'
\end{verbatim}
\afterverb
%
A pesar de que no es necesario, resulta corriente encerrar las tuplas entre
paréntesis, lo que ayuda a identificarlas rápidamente dentro del
código en Python.

\index{paréntesis!en tuplas}

\beforeverb
\begin{verbatim}
>>> t = ('a', 'b', 'c', 'd', 'e')
\end{verbatim}
\afterverb
%
Para crear una tupla con un único elemento, es necesario incluir una coma
al final:

\index{singleton}
\index{tupla!singleton}

\beforeverb
\begin{verbatim}
>>> t1 = ('a',)
>>> type(t1)
<type 'tuple'>
\end{verbatim}
\afterverb
%
Sin la coma, Python trata \verb"('a')" como una expresión con una
cadena dentro de un paréntesis, que evalúa como de tipo ``string'':

\beforeverb
\begin{verbatim}
>>> t2 = ('a')
>>> type(t2)
<type 'str'>
\end{verbatim}
\afterverb
%
Otro modo de construir una tupla es usar la función interna {\tt tuple}.
Sin argumentos, crea una tupla vacía:

\index{tupla, función}
\index{función!tupla}

\beforeverb
\begin{verbatim}
>>> t = tuple()
>>> print t
()
\end{verbatim}
\afterverb
%
Si el argumento es una secuencia (cadena, lista o tupla), el resultado
de la llamada a {\tt tuple} es una tupla con los elementos de la secuencia:

\beforeverb
\begin{verbatim}
>>> t = tuple('altramuces')
>>> print t
('a','l', 't', 'r', 'a', 'm', 'u', 'c', 'e', 's')
\end{verbatim}
\afterverb
%
Dado que {\tt tuple} es el nombre de un constructor, debe evitarse
el utilizarlo como nombre de variable.

La mayoría de los operadores de listas funcionan también con tuplas. El operador corchete
indexa un elemento:

\index{corchete, operador}
\index{operador!corchete}

\beforeverb
\begin{verbatim}
>>> t = ('a', 'b', 'c', 'd', 'e')
>>> print t[0]
'a'
\end{verbatim}
\afterverb
%
Y el operador de rebanada selecciona un rango de elementos.

\index{rebanada, operador}
\index{operador!slice}
\index{operador!rebanada}
\index{tupla!rebanada}
\index{rebanada!tupla}

\beforeverb
\begin{verbatim}
>>> print t[1:3]
('b', 'c')
\end{verbatim}
\afterverb
%
Pero si se intenta modificar uno de los elementos de la tupla, se
obtiene un error:

\index{exception!TypeError}
\index{TypeError}
\index{elemento, asignación}
\index{asignación!elemento}

\beforeverb
\begin{verbatim}
>>> t[0] = 'A'
TypeError: object doesn't support item assignment
\end{verbatim}
\afterverb
%
No se pueden modificar los elementos de una tupla, pero se puede
reemplazar una tupla con otra:

\beforeverb
\begin{verbatim}
>>> t = ('A',) + t[1:]
>>> print t
('A', 'b', 'c', 'd', 'e')
\end{verbatim}
\afterverb
%

\section{Comparación de tuplas}

\index{comparación!tupla}
\index{tupla!comparación}
\index{sort, método}
\index{método!sort}

Los operadores de comparación funcionan también con las tuplas y otras secuencias.
Python comienza comparando el primer elemento de cada
secuencia. Si es igual en ambas, pasa al siguiente elemento,
y así sucesivamente, hasta que encuentra uno que es diferente. A partir de ese momento,
los elementos siguientes ya no son tenidos en cuenta (aunque sean muy grandes).


\beforeverb
\begin{verbatim}
>>> (0, 1, 2) < (0, 3, 4)
True
>>> (0, 1, 2000000) < (0, 3, 4)
True
\end{verbatim}
\afterverb
%
La función {\tt sort} funciona del mismo modo. En principio
ordena por el primer elemento, pero en caso de que haya dos iguales,
usa el segundo, y así sucesivamente. 

Esta característica se presta al uso de un diseño llamado {\bf DSU}, que

\begin{description}

\item[Decorate] (Decora) una secuencia, construyendo una lista de tuplas
con uno o más índices ordenados precediendo los elementos de dicha secuencia,

\item[Sort] (Ordena) la lista de tuplas usando la función incorporada en Python {\tt sort}, y

\item[Undecorate] (Quita la decoración), extrayendo los elementos ordenados de la secuencia.

\end{description}

\label{DSU}
\index{DSU, diseño}
\index{diseño!DSU}
\index{decorate-sort-undecorate pattern}
\index{decorar-ordenar-quitar la decoración, diseño}
\index{diseño!decorate-sort-undecorate}
\index{Romeo and Juliet}

Por ejemplo, supón que tienes una lista de palabras y que quieres
ordenarlas de más larga a más corta:

\beforeverb
\begin{verbatim}
txt = 'Pero qué luz se deja ver allí'
palabras = txt.split()
t = list()
for palabra in palabras:
   t.append((len(palabra), palabra))

t.sort(reverse=True)

res = list()
for longitud, palabra in t:
    res.append(palabra)

print res
\end{verbatim}
\afterverb
%
El primer bucle crea una lista de tuplas, en la que cada
tupla es la palabra precedida por su longitud.

{\tt sort} compara el primer elemento (longitud), y
sólo tiene en cuenta el segundo en caso de empate. El argumento
clave {\tt reverse=True} indica a {\tt sort} que debe ir en orden decreciente.

\index{palabra clave, argumento}
\index{argumento!palabra clave}
\index{recorrido}

El segundo bucle recorre la lista de tuplas y crea una lista de
palabras en orden descendente según su longitud. Las palabras con cuatro caracteres,
por ejemplo, son ordenadas en orden alfabético {\em inverso}, de modo que
``deja'' aparece antes que ``allí'' en esa lista.

La salida del programa es la siguiente:
%
\beforeverb
\begin{verbatim}
['deja', 'allí', 'Pero', 'ver', 'qué', 'luz', 'se']
\end{verbatim}
\afterverb
%
Por supuesto, la línea pierde mucho de su impacto poético
cuando la convertimos en una lista de Python y la ordenamos
en orden descendente según la longitud de sus palabras.

\section{Asignación de tuplas}
\label{tuple assignment}

\index{tupla!asignación}
\index{asignación!tupla}
\index{intercambio, patrón}
\index{patrón!de intercambio}

Una de las características sintácticas del lenguaje Python que resulta única
es la capacidad de tener una tupla en el lado
izquierdo de una sentencia de asignación. Esto permite asignar
varias variables el mismo tiempo cuando tenemos una secuencia
en el lado izquierdo.

En este ejemplo tenemos una lista de dos elementos (por lo que se trata de una secuencia), y
asignamos los elementos primero y segundo de la secuencia
a las variables {\tt x} e {\tt y} en una única sentencia.

\beforeverb
\begin{verbatim}
>>> m = [ 'pásalo', 'bien' ]
>>> x, y = m
>>> x
'pásalo'
>>> y
'bien'
>>> 
\end{verbatim}
\afterverb
%
No es magia, Python traduce \emph{aproximadamente} la
sintaxis de asignación de la tupla
a lo siguiente:\footnote{Python no convierte la
sintaxis de forma literal. Por ejemplo, si intentas esto con un diccionario,
no funcionará como esperarías.}

\beforeverb
\begin{verbatim}
>>> m = [ 'pásalo', 'bien' ]
>>> x = m[0]
>>> y = m[1]
>>> x
'pásalo'
>>> y
'bien'
>>> 
\end{verbatim}
\afterverb

Estilísticamente, cuando usamos una tupla en el lado izquierdo de la sentencia
de asignación, omitimos los paréntesis. Pero lo que se muestra a continuación
es una sintaxis igualmente válida:

\beforeverb
\begin{verbatim}
>>> m = [ 'pásalo', 'bien' ]
>>> (x, y) = m
>>> x
'pásalo'
>>> y
'bien'
>>> 
\end{verbatim}
\afterverb
%
Una aplicación especialmente ingeniosa de asignación usando una tupla nos
permite {\bf intercambiar} los valores de dos variables en una única sentencia:

\beforeverb
\begin{verbatim}
>>> a, b = b, a
\end{verbatim}
\afterverb
%
Ambos lados de esta sentencia son tuplas, pero
el lado izquierdo es una tupla de variables; el lado derecho es una tupla
de expresiones. Cada valor en el lado derecho
es asignado a su respectiva variable en el lado izquierdo.
Todas las expresiones en el lado derecho son evaluadas antes de
realizar ninguna asignación.

El número de variables en el lado izquierdo y el número de
valores en el derecho deben ser el mismo:

\index{exception!ValueError}
\index{ValueError}

\beforeverb
\begin{verbatim}
>>> a, b = 1, 2, 3
ValueError: too many values to unpack
\end{verbatim}
\afterverb
%
Generalizando más, el lado derecho puede ser cualquier tipo de secuencia
(cadena, lista o tupla). Por ejemplo, para dividir una dirección de e-mail
en nombre de usuario y dominio, podrías escribir:

\index{split, método}
\index{método!split}
\index{dirección email}

\beforeverb
\begin{verbatim}
>>> dir = 'monty@python.org'
>>> nombreus, dominio = dir.split('@')
\end{verbatim}
\afterverb
%
El valor de retorno de {\tt split} es una lista con dos elementos;
el primer elemento es asignado a {\tt nombreus}, el segundo a
{\tt dominio}.

\beforeverb
\begin{verbatim}
>>> print nombreus
monty
>>> print dominio
python.org
\end{verbatim}
\afterverb
%

\section{Diccionarios y tuplas}

\index{diccionario}
\index{items, método}
\index{método!items}
\index{pareja clave-valor}

Los diccionarios tienen un método llamado {\tt items} que devuelve una lista de
tuplas, cada una de las cuales es una pareja clave-valor
\footnote{Este comportamiento es ligeramente diferente en Python 3.0.}.

\beforeverb
\begin{verbatim}
>>> d = {'a':10, 'b':1, 'c':22}
>>> t = d.items()
>>> print t
[('a', 10), ('c', 22), ('b', 1)]
\end{verbatim}
\afterverb
%
Como sería de esperar en un diccionario, los elementos no
tienen ningún orden en particular.

Sin embargo, dado que la lista de tuplas es una lista, y las tuplas
son comparables, ahora podemos ordenar la lista de tuplas. Convertir un diccionario
en una lista de tuplas es un método para obtener el contenido de un
diccionario ordenado según sus claves:

\beforeverb
\begin{verbatim}
>>> d = {'a':10, 'b':1, 'c':22}
>>> t = d.items()
>>> t
[('a', 10), ('c', 22), ('b', 1)]
>>> t.sort()
>>> t
[('a', 10), ('b', 1), ('c', 22)]
\end{verbatim}
\afterverb
%
La nueva lista está ordenada alfabéticamente en orden ascendente según el valor de sus claves.

\section{Asignación múltiple con diccionarios}

\index{recorrido!diccionario}
\index{diccionario!recorrido}

La combinación de {\tt items}, asignación en tupla y {\tt for},
consigue un bonito diseño de código para recorrer las claves y valores
de un diccionario en un único bucle:

\beforeverb
\begin{verbatim}
for clave, valor in d.items():
    print valor, clave
\end{verbatim}
\afterverb
%
Este bucle tiene dos {\bf variables de iteración}, ya que {\tt items} devuelve
una lista de tuplas y {\tt clave, valor} es una asignación en tupla,
que itera sucesivamente a través de cada una de las parejas clave-valor del
diccionario.

Para cada iteración a través del bucle, tanto {\tt clave} como {\tt valor} van
pasando a la siguiente pareja clave-valor del diccionario
(aquí también en orden de dispersión).

La salida de este bucle es:

\beforeverb
\begin{verbatim}
10 a
22 c
1 b
\end{verbatim}
\afterverb
%
Otra vez obtenemos un orden de dispersión (es decir, ningún orden concreto).

Si combinamos estas dos técnicas, podemos imprimir el contenido
de un diccionario ordenado por el \emph{valor} almacenado en cada pareja
clave-valor.

Para conseguirlo, primero creamos una lista de tuplas, donde cada tupla es
{\tt (valor, clave)}. El método {\tt items} nos dará una lista de
tuplas {\tt (clave, valor)}---pero esta vez queremos ordenar por valor, no
por clave. Una vez que hayamos construido la lista con las tuplas clave-valor,
resulta sencillo ordenar la lista en orden inverso e imprimir la nueva lista ordenada.

\beforeverb
\begin{verbatim}
>>> d = {'a':10, 'b':1, 'c':22}
>>> l = list()
>>> for clave, valor in d.items() :
...     l.append( (valor, clave) )
... 
>>> l
[(10, 'a'), (22, 'c'), (1, 'b')]
>>> l.sort(reverse=True)
>>> l
[(22, 'c'), (10, 'a'), (1, 'b')]
>>> 
\end{verbatim}
\afterverb
%
Al construir la lista de tuplas, hay que tener la precaución de colocar el valor
como primer elemento de cada tupla, de modo que luego podamos ordenar la lista de tuplas
y así obtener el contenido de nuestro diccionario ordenado por valor.

\section{Las palabras más comunes}

\index{Romeo and Juliet}
Volviendo a nuestro ejemplo anterior del texto de \emph{Romeo and Juliet}
Acto 2, Escena 2, podemos mejorar nuestro programa para hacer uso de esta técnica e
imprimir las diez palabras más comunes en el texto, como vemos a continuación:

\beforeverb
\begin{verbatim}
import string
manf = open('romeo-full.txt')
contadores = dict()
for linea in manf:
    linea = linea.translate(None, string.punctuation)
    linea = linea.lower()
    palabras = linea.split()
    for palabra in palabras:
        if palabra not in contadores:
            contadores[palabra] = 1
        else:
            contadores[palabra] += 1

# Ordenar el diccionario por valor
lst = list()
for clave, valor in contadores.items():
    lst.append( (valor, clave) )

lst.sort(reverse=True)

for clave, valor in lst[:10] :
    print clave, valor
\end{verbatim}
\afterverb
%
La primera parte del programa, que lee el archivo y procesa
el diccionario mapeando cada palabra en el contador de palabras del
documento, no ha cambiado. Pero en lugar de simplemente imprimir en pantalla
{\tt contadores} y terminar el programa, ahora construimos una
lista de tuplas {\tt (valor, clave)} y luego ordenamos la lista en orden inverso.

Dado que el valor va primero, se utilizará para las comparaciones.
Si hay más de una tupla con el mismo valor, se tendrá en cuenta
el segundo elemento (la clave), de modo que las tuplas cuyo valor sea
el mismo serán además ordenadas alfabéticamente según su clave.

Al final escribimos un bonito bucle {\tt for} que hace una iteración con
asignación múltiple e imprime en pantalla las diez palabras más comunes,
iterando a través de una rebanada de la lista ({\tt lst[:10]}).

De modo que la salida al final se parece a lo que queríamos para
nuestro análisis de frecuencia de palabras.

\beforeverb
\begin{verbatim}
61 i
42 and
40 romeo
34 to
34 the
32 thou
32 juliet
30 that
29 my
24 thee
\end{verbatim}
\afterverb
%
El hecho de que este complejo procesado y análisis de datos
puedan ser realizado con un programa Python de 19 líneas
sencillo de entender, es una de las razones por las que Python es una buena elección
como lenguaje para explorar información.

\section{Uso de tuplas como claves en diccionarios}

\index{tupla!como clave en diccionario}
\index{hashable}
\index{dispersable}

Dado que las tuplas son {\bf hashables} (dispersables) y las listas no, si queremos
crear una clave {\bf compuesta} para usar en un diccionario, deberemos usar una tupla
como clave.

Usaríamos por ejemplo una clave compuesta si quisiésemos crear un
directorio telefónico que mapease
parejas apellido, nombre con números de teléfono. Asumiendo
que hemos definido las variables
{\tt apellido}, {\tt nombre}, y {\tt numero}, podríamos escribir
una sentencia de asignación de diccionario como la siguiente:

\beforeverb
\begin{verbatim}
directorio[apellido,nombre] = numero
\end{verbatim}
\afterverb
%
La expresión dentro de los corchetes es una tupla. Podríamos usar
asignaciones mediante tuplas en un bucle {\tt for} para recorrer este diccionario.

\index{tupla!en corchetes}

\beforeverb
\begin{verbatim}
for apellido, nombre in directorio:
    print nombre, apellido, directorio[apellido, nombre]
\end{verbatim}
\afterverb
%
Este bucle recorre las claves de {\tt directorio}, que son tuplas.
Asigna los elementos de cada tupla a {\tt apellido} y {\tt nombre}, luego
imprime el nombre, apellido y número de teléfono correspondiente.

\section{Secuencias: cadenas, listas, y tuplas---¡Oh, Dios mío!}
\index{secuencia}

Me he centrado en las listas y tuplas, pero casi todos los ejemplos de
este capítulo funcionan también en listas de listas, tuplas de tuplas y
tuplas de listas. Para evitar enumerar todas las combinaciones posibles,
a veces resulta más sencillo hablar de secuencias de secuencias.

En muchos contextos, los diferentes tipos de secuencias (cadenas, listas, y
tuplas) pueden intercambiarse. De modo que, ¿cuándo y por qué elegir uno
u otro?

\index{cadena}
\index{lista}
\index{tupla}
\index{mutabilidad}
\index{inmutabilidad}

Para comenzar con lo más obvio, las cadenas están más limitadas que las demás
secuencias, porque los elementos deben ser caracteres. También son
inmutables. Si necesitas la capacidad de cambiar los caracteres
en una cadena (en vez de crear una nueva), puede que
lo más adecuado sea elegir una lista de caracteres.

Las listas se usan con más frecuencia que las tuplas, principalmente porque son mutables.
Pero hay algunos pocos casos donde es posible que prefieras usar las tuplas:

\begin{enumerate}

\item En algunos contextos, como una sentencia {\tt return}, resulta
sintácticamente más simple crear una tupla que una lista. En otros
contextos, es posible que prefieras una lista.

\item Si quieres usar una secuencia como una clave en un diccionario,
debes usar un tipo inmutable como una tupla o una cadena.

\item Si estás pasando una secuencia como argumento de una función,
el uso de tuplas reduce los comportamientos potencialmente indeseados
debido a la creación de alias.

\end{enumerate}

Dado que las tuplas son inmutables, no proporcionan métodos
como {\tt sort} y {\tt reverse}, que modifican listas ya existentes.
Sin embargo, Python proporciona las funciones integradas {\tt sorted}
y {\tt reversed}, que toman una secuencia como parámetro
y devuelven una secuencia nueva con los mismos elementos en un
orden diferente.

\index{sorted, función}
\index{función!sorted}
\index{reversed, función}
\index{función!reversed}


\section{Depuración}

\index{depuración}
\index{estructura de datos}
\index{forma, error de}
\index{error!forma}

Las listas, diccionarios y tuplas son conocidas de forma genérica como
{\bf estructuras de datos}; en este capítulo estamos comenzando a ver
estructuras de datos compuestas, como listas o tuplas, y diccionarios que contienen tuplas
como claves y listas como valores. Las estructuras de datos compuestas son útiles, pero
también resultan propensas a lo que yo llamo {\bf errores de modelado}; es decir, errores
causados cuando una estructura de datos tiene el tipo, tamaño o composición incorrecto,
o tal vez al escribir una parte del código se nos olvidó cual era el modelado
de los datos y se introdujo un error.

Por ejemplo, si estás esperando una lista con un entero y te
paso simplemente un entero sin más (no en una lista), no funcionará.

Cuando estés depurando un programa, y especialmente si estás
trabajando en un fallo complicado, hay cuatro cosas que puedes probar:

\begin{description}

\item[lectura:] Examina tu código, léelo para ti, y comprueba
si en realidad dice lo que querías que dijera.

\item[ejecución:] Experimenta haciendo cambios y ejecutando versiones
diferentes. A menudo, si muestras las cosas correctas en los lugares
adecuados del programa el problema se convierte en obvio, pero otras veces
tendrás que invertir algún tiempo construyendo unas ciertas estructuras.

\item[rumiado:] ¡Tómate tu tiempo para reflexionar! ¿De qué tipo de error
se trata: sintáctico, de ejecución, semántico? ¿Qué información puedes obtener de
los mensajes de error, o de la salida del programa? ¿Qué tipo de
error podría causar el problema que estás viendo? ¿Qué fue lo último
que cambiaste, antes de que el problema apareciera?

\item[retirada:] En algunos casos, lo mejor que se puede hacer
es dar marcha atrás, deshaciendo los últimos cambios, hasta llegar
a un punto en que el programa funcione y tú seas capaz de entenderlo. A partir de ahí,
puedes comenzar a reconstruirlo.

\end{description}

Los programadores novatos a veces se quedan atascados en una de estas actividades
y olvidan las otras. Cada actividad cuenta con su propio tipo
de fracaso.

\index{error!tipográfico}

Por ejemplo, leer tu código puede ayudarte si el problema es un
error tipográfico, pero no si se trata de un concepto
erróneo. Si no comprendes qué es lo que hace el programa, puedes
leerlo 100 veces y nunca encontrarás el error, porque el error está en
tu cabeza.

\index{depuración!experimental}

Hacer experimentos puede ayudar, especialmente si estás ejecutando
pruebas pequeñas y sencillas. Pero si ejecutas experimentos sin pararte a pensar
o leer tu código, puedes caer en el modelo que yo llamo ``sistema de programación al azar'',
que es el proceso de hacer cambios aleatorios hasta que el programa
hace lo que tiene que hacer. No es necesario decir que este tipo de programación
puede llevar mucho tiempo.

\index{programación al azar}
\index{plan de desarrollo!programación al azar}

Debes de tomarte tu tiempo para reflexionar. La depuración es como una
ciencia experimental. Debes tener al menos una hipótesis acerca
de dónde está el problema. Si hay dos o más posibilidades, intenta
pensar en una prueba que elimine una de ellas.

Tomarse un respiro ayuda a pensar. También hablar.
Si explicas el problema a alguien más (o incluso a ti mismo),
a veces encontrarás la respuesta antes de haber terminado de hacer la pregunta.

Pero incluso las mejores técnicas de depurado pueden fallar si hay demasiados
errores, o si el código que se está intentando arreglar es demasiado grande
y complicado. A veces la mejor opción es retirarse y simplificar el
programa hasta tener algo que funcione y que se sea
capaz de entender.

Los programadores novatos a menudo se muestran reacios a volver atrás,
no pueden tolerar la idea de borrar ni una línea de código (incluso si está mal).
Si eso te hace sentirte mejor, puedes copiar tu programa en otro archivo
antes de empezar a eliminar cosas. Luego podrás volver a pegar los
trozos poco a poco.

Encontrar un fallo difícil requiere leer, ejecutar, rumiar, y
a veces, retirarse. Si te quedas atascado en una de estas actividades,
intenta pasar a una de las otras.


\section{Glosario}

\begin{description}

\item[asignación en tupla:] Una asignación con una secuencia en el
lado derecho y una tupla de variables en el izquierdo. Primero
se evalúa el lado derecho y luego sus elementos son asignados a las
variables de la izquierda.
\index{tupla!asignación en}
\index{asignación!tupla}

\item[comparable:] Un tipo en el cual un valor puede ser contrastado para ver si es
mayor que, menor que, o igual que otro valor del mismo tipo.
Los tipos que son comparables pueden ser puestos en una lista y ordenados.
\index{comparable}

\item[estructura de datos:] Una colección de valores relacionados, a menudo
organizados en listas, diccionarios, tuplas, etc.
\index{estructura de datos}

\item[DSU:] Abreviatura de ``decorate-sort-undecorate
(decorar-ordenar-quitar la decoración)'',
un patrón que implica construir una lista de tuplas, ordenar, y
extraer parte del resultado.
\index{DSU, diseño}

\item[hashable (dispersable):] Un tipo que tiene una función de dispersión. Los tipos
inmutables, como enteros,
flotantes y cadenas son hashables (dispersables); los tipos mutables como listas y
diccionarios no lo son.
\index{hashable}
\index{dispersable}

\item[dispersar:] La operación de tratar una secuencia como una lista de
argumentos.
\index{dispersar}

\item[modelado (de una estructura de datos):] Un resumen del tipo,
tamaño, y composición de una estructura de datos.
\index{modelado}

\item[reunir:] La operación de montar una tupla
como argumento de longitud variable.
\index{reunir}

\item[singleton:] Una lista (u otra secuencia) con un único elemento.
\index{singleton}

\item[tupla:] Una secuencia inmutable de elementos.
\index{tupla}

\end{description}


\section{Ejercicios}

\begin{ex}
Revisa el ejercicio 9.3, del tema anterior, de este modo:
Lee y procesa las líneas ``From'' y extrae la
dirección. Cuenta el número de
mensajes de cada persona usando un diccionario.
	
Después de que todos los datos hayan sido leídos, para mostrar
la persona con más envíos, crea
una lista de tuplas (contador, email) desde el
diccionario. Luego ordena la lista en orden
inverso y muestra la persona que tiene más
envíos.

\beforeverb
\begin{verbatim}
Línea de ejemplo:
From stephen.marquard@uct.ac.za Sat Jan  5 09:14:16 2008

Introduce un nombre de fichero: mbox-short.txt
cwen@iupui.edu 5

Introduce un nombre de fichero: mbox.txt
zqian@umich.edu 195
\end{verbatim}
\afterverb
\end{ex}
\begin{ex}
Crea un programa que cuente la distribución de las horas del día para
cada uno de los mensajes. Puedes extraer la hora de la línea
``From'', buscando la cadena horaria y luego dividiendo esa cadena
en partes mediante el carácter dos-puntos. Una vez que tengas acumulados
los contadores para cada hora, imprime en pantalla los contadores, uno por línea,
ordenados por hora como se muestra debajo.
\beforeverb
\begin{verbatim}
Ejecución de ejemplo:
python timeofday.py
Introduce un nombre de fichero: mbox-short.txt
04 3
06 1
07 1
09 2
10 3
11 6
14 1
15 2
16 4
17 2
18 1
19 1
\end{verbatim}
\afterverb
\end{ex}


\begin{ex}
Escribe un programa que lea un archivo e
imprima las {\em letras} en orden decreciente de frecuencia. El programa
debe convertir todas las entradas a minúsculas y contar sólo las letras a-z.
El programa no debe contar espacios, dígitos, signos de puntuación, ni nada
que sea distinto a las letras a-z.
Busca ejemplos de texto de varios idiomas diferentes, y observa cómo la frecuencia
de las letras es diferente en cada idioma. Compara tus resultados con las tablas de
\url{wikipedia.org/wiki/Letter_frequencies}.

\index{letras, frecuencia de}
\index{frecuencia!letras}

\end{ex}