% LaTeX source for ``Python for Informatics: Exploring Information''
% Copyright (c)  2010-  Charles R. Severance, All Rights Reserved

\chapter{Diccionarios}
\index{dictionary}

\index{dictionary}
\index{type!dict}
\index{key}
\index{key-value pair}
\index{index}

Un {\bf diccionario} es similar a una lista, pero más general. En una lista,
las posiciones de los índices deben ser enteros; en un diccionario,
los índices pueden ser de (casi) cualquier tipo.

Puedes pensar en un diccionario como un mapa entre un conjunto de índices
(a los cuales se les llama {\bf claves}) y un conjunto de valores. Cada clave apunta a un
valor. La asocicación de una clave y un valor recibe el nombre de {\bf
par clave-valor}, o a veces {\bf elemento}.

Por ejemplo, hemos construido un diccionario que mapea desde palabras inglesas
a sus equivalentes en español, de modo que tanto claves como valores son cadenas.

La función {\tt dict} crea un diccionario nuevo sin elementos.
Dado que {\tt dict} es el nombre de una función incorporada, debes
evitar usarla como nombre de variable.

\index{dict function}
\index{function!dict}

\beforeverb
\begin{verbatim}
>>> eng2sp = dict()
>>> print eng2sp
{}
\end{verbatim}
\afterverb

Las llaves \verb"{}", representan un diccionario vacío.
Para añadir elementos al diccionario, se pueden usar corchetes:

\index{squiggly bracket}
\index{bracket!squiggly}

\beforeverb
\begin{verbatim}
>>> eng2sp['one'] = 'uno'
\end{verbatim}
\afterverb
%
Esta línea crea un elemento con la clave {\tt 'one'}
que apunta al valor \verb"'uno'". Si imprimimos el
diccionario de nuevo, veremos un par clave-valor con dos-puntos
entre la clave y el valor:

\beforeverb
\begin{verbatim}
>>> print eng2sp
{'one': 'uno'}
\end{verbatim}
\afterverb
%
Este formato de salida es también un formato de entrada. Por ejemplo,
puedes crear un nuevo diccionario con tres elementos:

\beforeverb
\begin{verbatim}
>>> eng2sp = {'one': 'uno', 'two': 'dos', 'three': 'tres'}
\end{verbatim}
\afterverb
%
Pero si ahora imprimes {\tt eng2sp}, puedes encontrarte con una sorpresa:

\beforeverb
\begin{verbatim}
>>> print eng2sp
{'one': 'uno', 'three': 'tres', 'two': 'dos'}
\end{verbatim}
\afterverb
%
El orden de los pares clave-valor no es el mismo. De hecho, si
escribes el mismo ejemplo en tu ordenador, puedes obtener un
resultado diferente. En general, el orden de los elementos en
un diccionario es impredecible.

Pero eso no es un problema, porque
los elementos de un diccionario nunca son indexados por índices enteros.
En lugar de eso, se usan las claves para buscar los valores correspondientes:

\beforeverb
\begin{verbatim}
>>> print eng2sp['two']
'dos'
\end{verbatim}
\afterverb
%
La clave {\tt 'two'} siempre apunta al valor \verb"'dos'", de modo que el orden
de los elementos no importa.

Si la clave especificada no está en el diccionario, se obtiene una excepción:

\index{exception!KeyError}
\index{KeyError}

\beforeverb
\begin{verbatim}
>>> print eng2sp['four']
KeyError: 'four'
\end{verbatim}
\afterverb
%
La función {\tt len} funciona en los diccionarios; devuelve el
número de parejas clave-valor:

\index{len function}
\index{function!len}

\beforeverb
\begin{verbatim}
>>> len(eng2sp)
3
\end{verbatim}
\afterverb
%
El operador {\tt in} también funciona en los diccionarios; te dice si
algo aparece como \emph{clave} en el diccionario (que aparezca
como valor no sirve).

\index{membership!dictionary}
\index{in operator}
\index{operator!in}

\beforeverb
\begin{verbatim}
>>> 'one' in eng2sp
True
>>> 'uno' in eng2sp
False
\end{verbatim}
\afterverb
%
Para ver si algo aparece como valor en un diccionario, se
puede usar el método {\tt values}, que devuelve los valores como
una lista, y después usar el operador {\tt in} sobre esa lista:

\index{values method}
\index{method!values}

\beforeverb
\begin{verbatim}
>>> vals = eng2sp.values()
>>> 'uno' in vals
True
\end{verbatim}
\afterverb
%
El operador {\tt in} utiliza algoritmos diferentes para las listas y
para los diccionarios. Para las listas, usa un algoritmo lineal de búsqueda.
Según la lista se hace más larga, el tiempo de búsqueda
crece en proporción directa a la longitud de la lista.
Para los diccionarios, Python usa un
algoritmo llamado {\bf tabla de dispersión}, que tiene una propiedad destacada--el
operador {\tt in} emplea la misma cantidad de tiempo sin importar cuántos
elementos haya en el diccionario. No explicaré
por qué las funciones de dispersión son tan mágicas,
pero puedes leer más acerca de ello en
\url{es.wikipedia.org/wiki/Tabla_hash}.

\index{hash table}

\begin{ex}
\label{wordlist2}

\index{set membership}
\index{membership!set}

Escribe un programa que lea las palabras de {\tt words.txt} y
las almacene como claves en un diccionario. No importa qué
valores tengan. Después puedes usar el operador {\tt in}
como un modo rápido de comprobar si una cadena está en el
diccionario.

\end{ex}


\section{Diccionario como conjunto de contadores}
\label{histogram}

\index{counter}

Supongamos que te han dado una cadena y quieres contar cuántas veces
aparece cada letra. Hay varias formas de hacerlo:

\begin{enumerate}

\item Podrías crear 26 variables, una para cada letra del
alfabeto. Después, podrías recorrer la cadena y, para cada
carácter, aumentar el contador correspondiente, probablemente
usando un condicional encadenado.

\item Podrías crear una lista con 26 elementos. Luego podrías
convertir cada carácter en un número (usando la función incorporada
{\tt ord}), usar el número como índice dentro de la lista, y aumentar
el contador apropiado.

\item Podrías crear un diccionario con los caracteres como claves
y contadores como sus valores correspondientes. La primera vez
que veas un carácter, añadirías un elemento al diccionario.
Después, aumentarías el valor del elemento ya existente.

\end{enumerate}

Todas estas opciones realiza el mismo cálculo, pero cada
una de ellas implementa la computación de un modo diferente.

\index{implementation}

Una {\bf implementación} es un modo de realizar un cálculo;
algunas implementaciones son mejores que otras. Por ejemplo,
una ventaja de las implementaciones del diccionario es que
necesitamos saber de antemano qué letras aparecerán en la cadena
y sólo tendremos que hacer sitios para las letras que vayan apareciendo.

Así es como podría ser el código:

\beforeverb
\begin{verbatim}
palabra = 'brontosaurio'
d = dict()
for c in palabra:
    if c not in d:
        d[c] = 1
    else:
        d[c] = d[c] + 1
print d
\end{verbatim}
\afterverb
%
En realidad estamos realizando un {\bf histograma}, que es un término
estadístico para un conjunto de contadores (o frecuencias).

\index{histogram}
\index{frequency}
\index{traversal}

El bucle {\tt for} recorre
la cadena. Cada vez que entra en el bucle, si el carácter {\tt c} no
está en el diccionario, creamos un nuevo elemento con la clave {\tt c} y el
valor inicial 1 (ya que hemos visto esa letra una vez). Si {\tt c} ya
está en el diccionario, incrementamos {\tt d[c]}.

\index{histogram}

Aquí está la salida del programa:

\beforeverb
\begin{verbatim}
{'a': 1, 'b': 1, 'o': 3, 'n': 1, 's': 1, 'r': 2, 'u': 1, 't': 1, 'i': 1}
\end{verbatim}
\afterverb
%
El histograma indica que las letras {\tt 'a'} y \verb"'b'"
aparecen una vez;  \verb"'o'" aparece tres, y así con las demás.

\index{get method}
\index{method!get}

Los diccionarios tienen un método llamado {\tt get} que tomas una clave
y un valor por defecto. Si la clave aparece en el diccionario,
{\tt get} devuelve el valor correspondiente; si no, devuelve
el valor por defecto. Por ejemplo:

\beforeverb
\begin{verbatim}
>>> contadores = { 'chuck' : 1 , 'annie' : 42, 'jan': 100}
>>> print contadores.get('jan', 0)
100
>>> print contadores.get('tim', 0)
0
\end{verbatim}
\afterverb
%
Podemos usar {\tt get} para escribir nuestro bucle de histograma de forma más concisa.
Ya que el método {\tt get} maneja automáticamente el caso de que la clave
no esté en el diccionario, podemos reducir cuatro líneas a una
y eliminar la sentencia {\tt if}

\beforeverb
\begin{verbatim}
palabra = 'brontosaurio'
d = dict()
for c in palabra:
    d[c] = d.get(c,0) + 1
print d
\end{verbatim}
\afterverb
%
El uso del método {\tt get} para simplificar este bucle de recuento
termina resultando un ``idioma'' usado en Python con mucha frecuencia, y
lo utilizaremos muchas veces en el resto del libro. Así que deberías
tomarte un respiro y comparar el bucle usando la sentencia {\tt if}
y el operador {\tt in} con el mismo bucle usando el método {\tt get}.
Hacen exactamente lo mismo, pero uno es más breve.
\index{idiom}

\section{Diccionarios y archivos}

Uno de los usos más comunes de un diccionario es contar la aparición
de palabras en un archivo con texto escrito.
Empecemos con un archivo muy sencillo de
palabras tomados del texto de \emph{Romeo and Juliet}.

Para el primer conjunto de ejemplos, usaremos una versión acortada y simplificada
del texto, sin signos de puntuación. Más tarde trabajaremos con el texto completo
de la escena, con puntuación incluida.

\beforeverb
\begin{verbatim}
But soft what light through yonder window breaks
It is the east and Juliet is the sun
Arise fair sun and kill the envious moon
Who is already sick and pale with grief
\end{verbatim}
\afterverb
%
Vamos a escribir un programa en Python para ir leyendo las líneas del archivo,
dividir cada línea en una lista de palabras, movernos en bucle a través de cada
palabra de la línea y contar el número de veces que aparece cada una, usando un diccionario.

\index{nested loops}
\index{loop!nested}
Verás que tenemos dos bucles {\tt for}. El bucle exterior va leyendo las
líneas del archivo, mientras que el interior va iterando a través de cada
una de las palabras de una línea concreta. Esto es un ejemplo
de un patrón llamdo {\bf bucles anidados}, ya que uno de los bucles
es el \emph{exterior}, y el otro es el \emph{interior}.

Debido a que el bucle interior ejecuta todas sus iteracciones cada vez
que el bucle exterior realiza una sola, imaginamos
que el bucle interior va iterando ``más rápido'' y que el exterior lo hace
más lentamente.

\index{Romeo and Juliet}
La combinación de los dos bucles anidados asegura que se cuentan
todas las palabras en cada línea del archivo de entrada.

\beforeverb
\begin{verbatim}
nombref = raw_input('Introduce el nombre del fichero: ')
try:
    manf = open(nombref)
except:
    print 'No se pudo abrir el archivo:', fname
    exit()

contadores = dict()
for linea in manf:
    palabras = linea.split()
    for palabra in palabras:
        if palabra not in contadores:
            contadores[palabra] = 1
        else:
            contadores[palabra] += 1

print contadores
\end{verbatim}
\afterverb
%
Cuando hacemos funcionar el programa, veremos un volcado en bruto
de todos los contadores sin ordenar.
(el archivo {\tt romeo.txt} está disponible en
\url{www.py4inf.com/code/romeo.txt})

\beforeverb
\begin{verbatim}
python count1.py 
Introduce el nombre del fichero: romeo.txt
{'and': 3, 'envious': 1, 'already': 1, 'fair': 1, 
'is': 3, 'through': 1, 'pale': 1, 'yonder': 1, 
'what': 1, 'sun': 2, 'Who': 1, 'But': 1, 'moon': 1, 
'window': 1, 'sick': 1, 'east': 1, 'breaks': 1, 
'grief': 1, 'with': 1, 'light': 1, 'It': 1, 'Arise': 1, 
'kill': 1, 'the': 3, 'soft': 1, 'Juliet': 1}
\end{verbatim}
\afterverb
%
Resulta un poco incómodo buscar a través del diccionario para encontrar
cuál es la palabra más común y su contador, de modo que necesitamos añadir un poco
más de código en Pyhton que nos proporcione la salida de modo que nos resulte más útil.

\section{Bucles y diccionarios}

\index{dictionary!looping with}
\index{looping!with dictionaries}
\index{traversal}

Si se utiliza un diccionario como secuencia
en una sentencia {\tt for}, ésta recorrerá todas
las claves del diccionario. Este bucle
imprime cada clave y su valor correspondiente:

\beforeverb
\begin{verbatim}
contadores = { 'chuck' : 1 , 'annie' : 42, 'jan': 100}
for clave in contadores:
    print clave, contadores[clave]
\end{verbatim}
\afterverb
%
Aquí tenemos lo que muestra como salida:

\beforeverb
\begin{verbatim}
jan 100
chuck 1
annie 42
\end{verbatim}
\afterverb
%
Otra vez vemos que la claves aparecen sin ningún orden en particular.

\index{idiom}
Podemos usar este patrón para poner en práctica las diversas expresiones de bucles
que hemos descrito antes. Por ejemplo, si queremos
encontrar todas las entradas de un diccionario que tengan un valor
de más de diez, podríamos escribir el siguiente código:

\beforeverb
\begin{verbatim}
contadores = { 'chuck' : 1 , 'annie' : 42, 'jan': 100}
for clave in contadores:
    if contadores[clave] > 10 :
        print clave, contadores[clave]
\end{verbatim}
\afterverb
%
El bucle {\tt for} itera a través de las
{\em claves} del diccionario, de modo que podemos
usar el operador índice para recuperar el
{\em valor} correspondiente
para cada clave.
Aquí podemos ver el aspecto de la salida:

\beforeverb
\begin{verbatim}
jan 100
annie 42
\end{verbatim}
\afterverb
%
Sólo se muestran las entradas con un valor superior a 10.

\index{keys method}
\index{method!keys}
Si se quieren imprimir las claves en orden alfabético, primero
se debe crear una lista de las claves del diccionario, usando el
método {\tt keys}, que está disponible en los objetos del tipo diccionario.
Luego, habrá que ordenar esa lista
e irse desplazando a través de la lista ordenada, buscando cada
clave e imprimiendo las parejas clave-valor en orden,
como se muestra a continuación:

\beforeverb
\begin{verbatim}
contadores = { 'chuck' : 1 , 'annie' : 42, 'jan': 100}
lst = counts.keys()
print lst
lst.sort()
for clave in lst:
    print clave, contadores[clave]
\end{verbatim}
\afterverb
%
Aquí vemos cómo queda la salida:

\beforeverb
\begin{verbatim}
['jan', 'chuck', 'annie']
annie 42
chuck 1
jan 100
\end{verbatim}
\afterverb
%
Primero se muestra la lista de claves sin ordenar que
obtenemos usando el método {\tt keys}. Luego podemos ver las
parejas clave-valor ya en orden, imprimidas desde el bucle {\tt for}.

\section{Procesado avanzado de texto}

\index{Romeo and Juliet}
En el ejemplo anterior, usando el fichero {\tt romeo.txt},
hemos creado un archivo tan simple como fuera posible, eliminando
manualmente todos los signos de puntuación. El texto real
tiene montones de esos signos, como se muestra a continuación:

\beforeverb
\begin{verbatim}
But, soft! what light through yonder window breaks?
It is the east, and Juliet is the sun.
Arise, fair sun, and kill the envious moon,
Who is already sick and pale with grief,
\end{verbatim}
\afterverb
%
Dado que la función de Python {\tt split} busca espacios y
trata las palabras como fichas separadas por esos espacios, trataríamos
las palabras ``soft!'' y ``soft'' como \emph{diferentes}, y se crearía
una entrada separada en el diccionario para cada una de ellas.

Además, dado que el archivo contiene palabras en mayúsculas, también se
trataría a ``who'' y ``Who'' como palabras diferentes, con contadores
distintos.

Podemos solventar ambos problemas usando los métodos
de cadena {\tt lower}, {\tt punctuation}, y {\tt translate}.
{\tt translate} es el más sutil de estos métodos.
Aquí está la documentación para {\tt translate}:

\verb"string.translate(s, table[, deletechars])"

\emph{Elimina todos los caracteres de s que hay en deletechars (si hay alguno),
y luego traduce los caracteres usando table, que debe ser una cadena
de 256-caracteres, que proporcione la traducción para cada
valor de carácter, indexado por su ordinal. Si la tabla es None,
entonces sólo se realizará el borrado de caracteres.}

Nosotros no especificaremos el valor de {\tt table}, pero usaremos
el parámetro {\tt deletechars} para eliminar todos los signos de puntuación.
Incluso dejaremos que sea el propio Python quien nos diga la lista de caracteres
que él considera ``signos de puntuación'':

\beforeverb
\begin{verbatim}
>>> import string
>>> string.punctuation
'!"#$%&\'()*+,-./:;<=>?@[\\]^_`{|}~'
\end{verbatim}
\afterverb
%
Hacemos las siguientes modificaciones a nuestro programa:

\beforeverb
\begin{verbatim}
import string                                          # Código nuevo

nombref = raw_input('Introduce el nombre del fichero: ')
try:
    manf = open(nombref)
except:
    print 'El fichero no se pudo abrir:', nombref
    exit()

contadores = dict()
for linea in nombref:
    linea = linea.translate(None, string.punctuation)    # Código nuevo
    linea = linea.lower()                                # Código nuevo
    palabras = linea.split()
    for palabra in palabras:
        if palabra not in palabras:
            contadores[palabra] = 1
        else:
            contadores[palabra] += 1

print contadores
\end{verbatim}
\afterverb
%
Usamos {\tt translate} para eliminar todos los signos de puntuación, y {\tt lower} para
forzar la línea a minúsculas. El resto del programa no se ha modificado.
Para Python 2.5 y anteriores, {\tt translate} no
acepta {\tt None} como primer parámetro, de modo que en ese caso
habría que usar el siguiente código para la llamada a translate:

\beforeverb
\begin{verbatim}
print a.translate(string.maketrans(' ',' '), string.punctuation
\end{verbatim}
\afterverb
%
Parte del aprendizaje del ``Arte de Python'' o ``Pensar Pythónicamente''
consiste en darse cuenta de que Python
a menudo tiene capacidades incorporadas para muchos problemas de análisis
de datos comunes. Con el tiempo, habrás visto suficiente código de ejemplo y leído
suficiente documentación para saber dónde buscar para localizar si alguien
ya ha escrito antes algo que pueda hacer tu trabajo mucho más sencillo.

Lo siguiente es una versión abreviada de la salida:

\beforeverb
\begin{verbatim}
Enter the file name: romeo-full.txt
{'swearst': 1, 'all': 6, 'afeard': 1, 'leave': 2, 'these': 2, 
'kinsmen': 2, 'what': 11, 'thinkst': 1, 'love': 24, 'cloak': 1, 
a': 24, 'orchard': 2, 'light': 5, 'lovers': 2, 'romeo': 40, 
'maiden': 1, 'whiteupturned': 1, 'juliet': 32, 'gentleman': 1, 
'it': 22, 'leans': 1, 'canst': 1, 'having': 1, ...}
\end{verbatim}
\afterverb
%
Buscar a través de esta salida resulta todavía pesado y podemos utilizar
a Python para que nos dé exactamente lo que buscamos. Pero para eso,
tendremos que aprender algo sobre las {\bf tuplas} de Python.
Retomaremos este ejemplo una vez que hayamos estudiado las tuplas.

\section{Depuración}
\index{debugging}

Según vayas trabajando con conjuntos de datos más grandes, se irá haciendo más
pesado depurar imprimiendo y comprobando los datos de forma manual. He aquí algunas
sugerencias para depurar conjuntos de datos grandes:

\begin{description}

\item[Reduce la entrada:] Si es posible, reduce el tamaño del
conjunto de datos. Por ejemplo, si el programa lee un archivo de texto,
comienza solamente con las primeras 10 líneas, o con el ejemplo más pequeño
que puedas encontrar. Puedes editar los mismos archivos, o (mejor) modificar el
programa de modo que lea sólo las primeras {\tt n} líneas.

Si hay un error, puedes reducir {\tt n} hasta el valor
más pequeño en el cual se manifieste el error, y luego irlo incrementando gradualmente
hasta que encuentres y corrijas los errores.

\item[Comprueba los resúmenes y tipos:] En vez de imprimir y comprobar el
conjunto de datos completo, considera imprimir resúmenes de los datos: por ejemplo,
el número de elementos en un diccionario o el total de una lista de números.

Una causa habitual de errores en tiempo de ejecución es un valor que no es del tipo
correcto. Para depurar este tipo de error, a menudo es suficiente con imprimir
el tipo del valor.

\item[Escribe auto-comprobaciones:] A veces puedes escribir código que compruebe
los errores automáticamente. Por ejemplo, si estás calculando la
media de una lista de números, puedes comprobar que el resultado no
es mayor que el elemento más grande de la lista, o menor que el más
pequeño. A eso se le llama ``sanity check (prueba de sensatez)'', porque detecta
resultados que son ``completamente ilógicos''.

\index{sanity check}
\index{consistency check}

Otro tipo de comprobación compara los resultados de dos cálculos
diferentes para ver si son consistentes. A esto se le llama una
``consistency check (prueba de consistencia)''.

\item[Haz que la salida quede bonita:] Dar formato a la salida de depuración
puede hacer más fácil localizar un error.

\end{description}

Una vez más, el tiempo que emplees construyendo el andamiaje puede reducir
el tiempo que emplearás luego depurando.

\index{scaffolding}

\section{Glosario}

\begin{description}

\item[bucles anidados:] Cuando hay uno o más bucles ``dentro'' de
otro bucle. El bucle interior se ejecuta completo cada vez que el exterior
se ejecuta una vez.
\index{nested loops}
\index{loop!nested}

\item[búsqueda (lookup):] Una operación en un diccionario que toma una clave
y encuentra su valor correspondiente.
\index{lookup}

\item[clave:] Un objeto que aparece en un diccionario como la
primera parte de una pareja clave-valor.
\index{key}

\item[diccionario:] Un mapeo de un conjunto de claves hacia sus
valores correspondientes.
\index{dictionary}

\item[elemento:] Otro nombre para una pareja clave-valor.
\index{item!dictionary}

\item[función de dispersión (hash function):] Una función usada por una tabla de dispersión
para calcular la localización de una clave.
\index{hash function}

\item[histograma:] Un conjunto de contadores.
\index{histogram}

\item[implementación:] Un modo de realizar un cálculo.
\index{implementation}

\item[pareja clave-valor:] La representación del mapeado desde
una clave hacia un valor.
\index{key-value pair}

\item[tabla de disperasión (hashtable):] El algoritmo usado para implementar
los diccionarios de Python.
\index{hashtable}

\item[valor:] Un objeto que aparece en un diccionario como la
segunda parte de una pareja clave-valor. Es más específico que
el uso que hacíamos antes de la palabra ``valor''.
\index{value}

\end{description}

\section{Ejercicios}

\begin{ex}
Escribe un programa que categorice cada mensaje de correo según
el día de la semana en que fue hecho el envío. Para lograrlo, busca
las líneas que comienzan por ``From'', luego localiza la
tercera palabra y mantén un contador actualizado de cada uno
de los días de la semana. Al final del programa imprime en pantalla
el contenido de tu diccionario (el orden no importa).

\beforeverb
\begin{verbatim}
Línea de ejemplo:
From stephen.marquard@uct.ac.za Sat Jan  5 09:14:16 2008

Ejemplo de Ejecución:
python dow.py
Intoduce un nombre de fichero: mbox-short.txt
{'Fri': 20, 'Thu': 6, 'Sat': 1}
\end{verbatim}
\afterverb
\end{ex}

\begin{ex}
Escribe un programa que lea a través de un registro de correo,
construye un histograma usando un diccionario para contar cuántos
mensajes han llegado desde cada dirección de correo,
e imprime el diccionario.

\beforeverb
\begin{verbatim}
Introduce nombre del fichero: mbox-short.txt
{'gopal.ramasammycook@gmail.com': 1, 'louis@media.berkeley.edu': 3, 
'cwen@iupui.edu': 5, 'antranig@caret.cam.ac.uk': 1, 
'rjlowe@iupui.edu': 2, 'gsilver@umich.edu': 3, 
'david.horwitz@uct.ac.za': 4, 'wagnermr@iupui.edu': 1, 
'zqian@umich.edu': 4, 'stephen.marquard@uct.ac.za': 2, 
'ray@media.berkeley.edu': 1}
\end{verbatim}
\afterverb
\end{ex}

\begin{ex}
Añade código al programa anterior para localizar quién
tiene más mensajes en el archivo.

Después de que los datos hayan sido leídos y el diccionario haya
sido creado, busca a través del diccionario usando un bucle máximo
(mira la Section~\ref{maximumloop})
para encontrar quién es el que tiene más
mensajes e imprime cuántos mensajes tiene esa persona.

\beforeverb
\begin{verbatim}
Introduce un nombre de archivo: mbox-short.txt
cwen@iupui.edu 5

Introduce un nombre de archivo: mbox.txt
zqian@umich.edu 195
\end{verbatim}
\afterverb
\end{ex}

\begin{ex}
Este programa guarda el nombre de dominio (en vez de la dirección)
desde donde se envió el mensaje, en vez de quién mandó el mensaje
(es decir, la dirección de correo completa). Al final
del programa, imprime en pantalla el contenido de tu diccionario.

\beforeverb
\begin{verbatim}
python schoolcount.py
Introduce un nombre de archivo: mbox-short.txt
{'media.berkeley.edu': 4, 'uct.ac.za': 6, 'umich.edu': 7, 
'gmail.com': 1, 'caret.cam.ac.uk': 1, 'iupui.edu': 8}
\end{verbatim}
\afterverb
\end{ex}

